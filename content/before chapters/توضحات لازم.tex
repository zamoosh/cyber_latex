%! Author = zamoosh
%! Date = 6/4/23

{\setstretch{0.5}
\section*{توضحات لازم}
\label{sec:توضحات لازم}
\addcontentsline{toc}{section}{توضحات لازم}{\protect\numberline{}}
}
\paragraph{}
\textbf{اخلاق دئونتولوژیک یا اخلاق واجب‌گرایانه \textenglish{\textbf{(Deontological Ethics)}}}
یک نظریهٔ اخلاقی است که بر ترکیب ویژگی‌های اخلاقی عمل و رعایت وظایف و اصول اخلاقی تمرکز دارد.
این نظریه بر ایدهٔ آن تأکید می‌کند که برخی از اعمال به طور ذاتی صحیح یا نادرست هستند، بدون توجه به پیامدهای آن‌ها.
در اخلاق واجب‌گرایانه، اخلاقیت یک عمل توسط نیت پشت آن و رعایت قوانین یا وظایف اخلاقی تعیین می‌شود.
این نظریه بر مفاهیمی مانند عدالت، انصاف و احترام به حقوق فردی تأکید می‌کند.
نظریه‌های اخلاقی واجب‌گرایانه شامل اخلاق کانتی و نظریهٔ فرمان الهی می‌شوند.

\paragraph{}
\textbf{اخلاق فضیلت‌گرا \textenglish{\textbf{(Virtue Ethics)}}}
اخلاق فضیلت‌گرا یک نظریهٔ اخلاقی است که بر توسعهٔ صفات فضیلت‌آمیز و اخلاقی در افراد تأکید دارد.
این نظریه بر ایدهٔ اینکه بودن یک فرد با اخلاق خوب برای رفتار اخلاقی اساسی است تمرکز دارد.
اخلاق فضیلت‌گرا کمتر بر قوانین یا وظایف خاص تأکید می‌کند و به جای آن بر تقویت و تمرین ویژگی‌های فضیلت‌آمیز مانند راستگویی، مهربانی، شجاعت و حکمت تأکید می‌کند.
تمرکز بر توسعه و عمل به این فضیلت‌ها به منظور اتخاذ تصمیمات اخلاقی صحیح و زندگی فضیلت.

\paragraph{}
\textbf{اخلاق فایده‌گرا \textenglish{\textbf{(Utilitarian Ethics)}}}
اخلاق فایده‌گرا، همچنین به عنوان نتیجه‌گرایی شناخته می‌شود، یک نظریهٔ اخلاقی است که اخلاقیت یک عمل را بر اساس پیامدها یا نتایج آن تعیین می‌کند.\     این نظریه به نگرشی می‌پردازد که عملی صحیح، عملی است که به حداکثر شادی کلی یا خوبی برای بیشترین تعداد افراد منجر می‌شود.
اخلاق فایده‌گرا بر اولویت دادن به حداکثر خوبی برای بیشترین تعداد و کاهش رنج یا آسیب تمرکز می‌کند.
این نظریه بر محاسبه و ارزیابی نتایج برای تعیین مسیر اخلاقی عمل تأکید دارد.


\newpage


\phantomsection