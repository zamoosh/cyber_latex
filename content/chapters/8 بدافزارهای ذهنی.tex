%! Author = zamoosh
%! Date = 6/13/23


\chapter{بدافزارهای ذهنی: الگوریتم‌ها و معماری انتخاب}
\label{ch:بدافزارهای ذهنی الگوریتم‌ها و معماری انتخاب}
\phantomsection

\begin{quote}
    ما نمی‌خواهیم از مردم بپرسیم که قرار است چه کاری انجام دهند.
    زیرا می‌دانیم که این امر چندان پیش‌بینی‌کننده نحوه اجرای یک تبلیغ نیست، زیرا افراد به مغز چپ خود می‌روند و بیش از حد شروع به فکر کردن می‌کنند.
    \\\\
    \textbf{کری کالینگ، مدیر بازاریابی در شرکت بازاریابی \textenglish{\textbf{«System 1 Group»}}}
    \newline
\end{quote}


{\setstretch{0.5}
\phantomsection
\section*{رسوایی داده های کمبریج آنالیتیکا}
\label{sec:رسوایی داده های کمبریج آنالیتیکا}
\addcontentsline{toc}{section}{رسوایی داده های کمبریج آنالیتیکا}{\protect\numberline{}}
رسوایی داده‌های کمبریج آنالیتیکا ریشه در سال 2010 داشت که \textenglish{\textbf{Facebook}} برنامه \\ \textenglish{\textbf{\mbox{Open Graph}}} خود را راه‌اندازی کرد. \textenglish{\textbf{Open Graph}} به توسعه دهندگان برنامه‌های شخص ثالث اجازه می‌دهد تا به اطلاعات شخصی کاربران \textenglish{\textbf{Facebook}} و همچنین همه‌ی داده‌های «دوستان» خود دسترسی داشته‌باشند. در سال 2013، «محقق دانشگاهی الکساندر کوگان»، در همکاری با شرکت بازاریابی و تجزیه و تحلیل داده‌ها، کمبریج آنالیتیکا، اپلیکیشنی به نام «این زندگی دیجیتال شماست» راه‌اندازی کرد. این اپلیکیشن از کاربران دعوت کرد تا در یک مسابقه شخصیت‌شناسی رایگان شرکت کنند و حدود 300000 کاربر \textenglish{\textbf{Facebook}} این کار را انجام‌دادند. این برنامه داده‌های مربوط به پروفایل‌های روان‌سنجی آن‌ها را از آزمون جمع‌آوری کرد (که پنج ویژگی شخصیتی بزرگ کاربران را اندازه‌گیری می‌کرد) اما همچنین به‌طور آزادانه داده‌های \textenglish{\textbf{Facebook}} را از همه‌ی دوستان آن‌ها جمع‌آوری کرد. کمبریج آنالیتیکا در تلاش بود تا مجموعه‌ای از رای‌دهندگان آمریکایی را تا حد امکان جمع‌آوری کند.
}

در سال 2015، اولین گزارش‌ها منتشر شد مبنی بر اینکه کمپین سیاسی تد کروز میلیون‌ها نفر از این پروفایل‌های روان‌سنجی را در تلاش برای کسب مزیت در انتخابش به مجلس سنای ایالات متحده تجزیه و تحلیل کرده است (افشاگری که کاملاً نامحبوب بود و منجر به تضمین‌هایی از سوی \textenglish{\textbf{Facebook}} و کمبریج آنالیتیکا که داده‌های مورد نظر حذف شده است).

با این حال، در سال 2018 اخبار منتشر شد مبنی بر اینکه داده‌های ده‌ها میلیون کاربر \textenglish{\textbf{Facebook}} (شاید به 87 میلیون نفر) توسط کمبریج آنالیتیکا جمع‌آوری‌شده و در انتخابات ریاست‌جمهوری آمریکا در سال 2016 توسط ستاد انتخاباتی دونالد ترامپ استفاده شده‌است.
بسیاری از این افراد تست شخصیت را انجام نداده بودند، اما از زمانی که یکی از دوستانشان شرکت کرده بود، محققان می‌توانستند آزادانه به داده‌های آن‌ها دسترسی داشته‌باشند.
سپس کمبریج آنالیتیکا داده‌های \textenglish{\textbf{Facebook}} را با سایر داده‌هایی که خریداری کرده بود و همچنین فهرست‌های انتخاباتی محلی ارجاع داد.
بنابراین، کمبریج آنالیتیکا توانست پرونده‌های گسترده‌ای را در مورد ده‌ها میلیون رای‌دهنده، از جمله ویژگی‌های جمعیتی، ویژگی‌های شخصیتی، شبکه‌های اجتماعی، تاریخچه خرید، لایک‌ها، عضویت در احزاب سیاسی و غیره جمع‌آوری کند.
«مک‌نامی» تخمین می‌زند که این پرونده‌ها در نهایت شامل حدود 13 درصد همه‌ی رای دهندگان واجد شرایط ایالات متحده بوده.

کمبریج آنالیتیکا از مشخصات رأی دهندگان خود برای کسب برتری در انتخاب دونالد ترامپ به عنوان رئیس جمهور ایالات متحده در سال 2016 و همچنین کمپین «خروج» رفراندوم برگزیت در بریتانیا استفاده کرد.
همانطور که \textenglish{\textbf{Cadwalladr}} توضیح می‌دهد، کمبریج آنالیتیکا از نتایج آزمایش و داده‌های \textenglish{\textbf{Facebook}} برای ساخت الگوریتمی استفاده کرد که می‌تواند پروفایل‌های فردی \textenglish{\textbf{Facebook}} را تجزیه و تحلیل کند و ویژگی های شخصیتی مرتبط با رفتار رأی‌گیری را تعیین کند.
این الگوریتم به‌ویژه مؤثر بود زیرا به دانشمندان داده اجازه می‌داد تا رأی‌دهندگان نوسان را شناسایی کنند و سپس آن‌ها را با تبلیغات و پیام‌های خاصی که به احتمال زیاد رأی آن‌ها را «تحریک» می‌کرد، هدف قرار دهند.
\newline
\newline


{\setstretch{0.5}
\phantomsection
\section*{معماری انتخابی و فناوری متقاعد کننده: "جعبه ای برای انسان مدرن"}
\label{sec:معماری انتخابی و فناوری متقاعد کننده: "جعبه ای برای انسان مدرن"}
\addcontentsline{toc}{section}{معماری انتخابی و فناوری متقاعد کننده: "جعبه ای برای انسان مدرن"}{\protect\numberline{}}
کمبریج آنالیتیکا از داده‌ها برای آموزش الگوریتم‌های توصیه و ترویج محتوای رسانه‌های اجتماعی «قادر به حرکت دادن افکار عمومی در مقیاس» استفاده‌کرد.
الگوریتم‌های کمبریج آنالیتیکا این کار را با دسته‌بندی خرد افراد در گروه‌هایی که با ویژگی‌های جمعیت‌شناختی، سیاسی و روان‌سنجی تعریف می‌شوند انجام می‌دهند: برای مثال، رای دهندگان زن محافظه کار اجتماعی در حومه‌های مرفهِ دی سی که فرزندانشان تحت تاثیر تعطیلی مدارس مرتبط با کووید قرار‌گرفته‌اند، رای دهندگان طبقه کارگر در کمربند زنگ زده که درازمدت کم کار هستند، بازنشستگان طبقه پایین از فلوریدا مرکزی که نگران افزایش هزینه‌های مراقبت‌های بهداشتی هستند.
هدف از این الگوریتم‌ها هدف قرار دادن «رای‌دهندگان نوسان» بسیار مورد علاقه است تا بتوانند رأی خود را در مناطق مهم میدان نبرد تحت‌تأثیر قرار‌دهند.
بیش از این، آن‌ها می‌توانند الگوریتم‌های خود را در زمان واقعی در گروه‌های متمرکز آموزش‌دهند و بهبود بخشند.
}

اگرچه \textenglish{\textbf{Facebook}} به دلیل نقض داده‌ها توسط کمیسیون تجارت فدرال 5 میلیارد دلار جریمه شد، خطرات آنچه به عنوان "فناوری متقاعد کننده" شناخته می‌شود بسیار فراتر از رسوایی داده‌های کمبریج آنالیتیکا است.
مربیان و روانشناسان برای سال‌ها نگرانی‌هایی را در مورد فناوری متقاعدکننده مطرح کرده‌اند، اما این تأثیر کمی بر صنایع بازاریابی مصرف‌کننده و سیاسی داشت.
این رسوایی بسیار بیشتر از نقض حریم خصوصی کاربران است.
همانطور که مک نامی بیان می‌کند، این در مورد این است که چگونه "داده‌های ما هوش‌مصنوعی را تغذیه می‌کند که هدف آن‌ها دستکاری توجه و رفتار کاربران بدون اطلاع یا تایید آن‌ها است".

در حالی که بازاریابی مبتنی بر گروه‌های جمعیتی سابقه طولانی دارد، این عمل با حجم عظیمی از داده‌های ایجاد شده توسط رسانه‌های اجتماعی افزایش یافته‌است.
به عنوان مثال، \textenglish{\textbf{Facebook}} گروه‌هایی به نام \textenglish{\textbf{"lookalikes"}} ایجاد کرده‌است که کاربران را به گروه‌هایی با پروفایل‌های مشابه طبقه‌بندی می‌کند تا به شرکت در هدف‌گیری خرد آن‌ها کمک کند.
«کریستوفر ویلی، دانشمند سابق داده که در کمبریج آنالیتیکا افشاگر شد»، اظهار می‌دارد که به کاربران محتوایی بر اساس گروه مشابه خود ارائه می‌شود که سایر کاربران نمی‌بینند.
این امر باعث ایجاد حباب‌های فیلتر شده و شکاف‌های اجتماعی را عمیق‌تر می‌کند.
او بیان می‌کند که «خط ظریفی بین الگوریتمی وجود دارد که شما را تعریف می‌کند تا نشان دهد واقعاً چه کسی هستید و الگوریتمی که شما را برای ایجاد یک پیش‌گویی خودشکوفایی از اینکه فکر می‌کند باید تبدیل شوید، تعریف می‌کند».

شبیه‌سازی‌ها و ریزهدف‌گذاری از قدرت علم‌داده برای درگیر‌شدن (هرچند بسیار مؤثرتر) در رویه‌ی قدیمی تبلیغات استفاده می‌کنند.
بازاریابان، برای شرکت‌های مصرف‌کننده و کمپین‌های سیاسی، به طور معمول محتوای اخلاقی و بسیار احساسی را ارائه می‌دهند که به سرعت در رسانه‌های اجتماعی پخش می‌شود.
در واقع، رسانه‌های اجتماعی و دیگر پلت‌فرم‌های آنلاین اکنون «منابع اولیه محرک‌های اخلاقی مرتبطی هستند که افراد در زندگی روزمره خود تجربه می‌کنند».
کسانی که از رسانه‌های اجتماعی برای تأثیرگذاری بر افکار عمومی استفاده می‌کنند، از یادگیری پاداش اجتماعی نیز استفاده می‌کنند (معمولاً به شکل «اشتراک‌گذاری»، «کلیک»، «لایک»، «فالوور» و دیگر اشکال تقویت‌کننده تعامل).
این رفتارها نه تنها بسیار پاداش دهنده هستند، بلکه می‌توانند توسط سیستم‌های یادگیری‌ماشین استخراج شوند تا رفتار آینده‌ی ما، دوستان و پیروان ما و گروه‌های مشابه ما را پیش‌بینی کنند.
فردی را که با محتوای آنلاین درگیر می‌شود مانند موش در جعبه اسکینر که اهرم پاداش را فشار می‌دهد تصور کردند و به این نتیجه رسیدند که رسانه‌های اجتماعی مانند "جعبه اسکینر برای انسان مدرن است".

مقایسه بین تعامل در رسانه‌های اجتماعی و یک موش آزمایشی در یک جعبه عمیق است.
در سال 2014، مطالعه ای با همکاری \textenglish{\textbf{Facebook}} و دانشگاه کرنل منتشر شد)این آزمایش شامل آزمایشگاه غذا و برند نیست، بلکه دپارتمان علوم ارتباطات و اطلاعات بود(.
محققان محتوای احساسی پست‌هایی را که کاربران \textenglish{\textbf{Facebook}} در فیدهای خبری خود دریافت می‌کردند، دستکاری کردند، به‌ویژه از افرادی که به آن‌ها اعتماد داشتند، مانند دوستان و کسانی که دنبال می‌کردند.
محققان می‌خواستند ببینند آیا محتوای عاطفی مثبت در مقابل منفی بر خلق و خوی پست‌های بعدی کاربران تأثیر می‌گذارد یا خیر (به عنوان مثال، آیا شواهدی از «سرایت عاطفی» در رسانه‌های اجتماعی وجود دارد یا خیر).
وجود داشت، اما این مهم‌ترین جنبه مطالعه نبود.
به شرکت کنندگان اطلاع داده‌نشد که از آن‌ها به عنوان موضوع تحقیق استفاده می‌شود.
در واقع، رضایت آن‌ها برای شرکت در تحقیقات تجربی هرگز دریافت‌نشد.
از آنجایی که داده‌ها توسط \textenglish{\textbf{Facebook}} جمع‌آوری شده‌بود، محققان حتی به دنبال تأیید هیئت بررسی اخلاق پژوهشی کورنل نبودند.
معلوم نیست اگر می‌گرفتند تایید می‌شدند.

هنگامی که اخبار عدم‌رضایت منتشر شد، واکنش‌های منفی علیه این مطالعه وجود داشت.
اصل اصلی اخلاق تحقیق مستلزم کسب رضایت آگاهانه از شرکت کنندگان در تحقیق است.
این اصل در زمینه اخلاق پزشکی ایجاد شد (به \hyperref[sec:کادر 1.8]{\mbox{کادر 1.8}} مراجعه کنید) اما از آن زمان به تمام زمینه‌های تحقیقات آکادمیک مربوط به موضوعات انسانی گسترش یافته است.
بسیاری از متخصصان اخلاق زیستی استدلال می‌کردند که این تحقیق «به‌طور فاحش» هیچ‌یک از اصول قانون یا اخلاقی را نقض نمی‌کند، و اگر این کار را انجام می‌داد، به این معناست که شیوه‌های استاندارد \textenglish{\textbf{Facebook}} نیز از نظر اخلاقی مشکوک هستند.
«کاترین فلیک» با این استدلال پاسخ داد که اصول اخلاقی شرکت‌هایی که به طور معمول کاربران را بدون اطلاع یا رضایت آن‌ها در معرض دستکاری آزمایشی و آزمایشی قرار می‌دهند، دقیقاً موضوع مورد بحث است.

در حالی که دانشمند اصلی در مطالعه سرایت عاطفی عذرخواهی کرد، این نقض اخلاقی رسوایی کمبریج آنالیتیکا را که به زودی دنبال می‌شود، پیش‌بینی کرد.
اساتید برجسته در دانشگاه‌های معتبری مانند کمبریج و هاروارد از توسعه الگوریتم کمبریج آنالیتیکا اطلاع داشتند و آن را هیجان‌انگیز و نوآورانه می‌دانستند.
به نظر می‌رسد که بحثی در مورد اخلاق تحقیق وجود نداشته‌است.
همانطور که «ویلی» می‌گوید، «با توجه به اینکه دانشمندان دانشگاه‌های برجسته جهان به من می‌گفتند در آستانه «انقلاب‌سازی» علوم اجتماعی هستیم، من حریص شده‌بودم و جنبه‌های تاریک کاری را که انجام می‌دادیم نادیده می‌گرفتم».


% ======================================== کادر 8.1
\begin{tcolorbox}[colback=gray!10,colframe=black,breakable]

    \phantomsection
    \section*{کادر 1.8}
    \label{sec:کادر 1.8}
    \begin{Large}
        \textbf{\mbox{اصول قانون نورنبرگ}}
    \end{Large}
    \newline
    \newline
    اصول نورنبرگ در مورد آزمایش انسان عبارتند از:

    \begin{description}[leftmargin=0.5cm,style=nextline]
        \item[۱] رضایت داوطلبانه سوژه انسانی کاملاً ضروری است.
        \item[۲] آزمایش باید به گونه ای باشد که نتایج مثمر ثمری برای صلاح جامعه داشته باشد، غیرقابل تهیه با روش ها یا وسایل مطالعه دیگر باشد و ماهیت تصادفی و غیر ضروری نداشته باشد.
        \item[۳] آزمایش باید به گونه ای طراحی و بر اساس نتایج آزمایش بر روی حیوانات و آگاهی از تاریخچه طبیعی بیماری یا سایر مشکلات مورد مطالعه باشد که نتایج پیش بینی شده انجام آزمایش را توجیه کند.
        \item[۴] آزمایش باید به گونه ای انجام شود که از همه رنج ها و آسیب های جسمی و روحی غیر ضروری جلوگیری شود.
        \newpage
        \item[۵] هیچ آزمایشی نباید در جایی انجام شود که دلیل پیشینی وجود داشته باشد که باور شود مرگ یا آسیب ناتوان کننده رخ خواهد داد.
        به جز، شاید، در آزمایش هایی که پزشکان تجربی نیز به عنوان سوژه خدمت می کنند.
        \item[۶] درجه خطری که باید متحمل شود هرگز نباید بیشتر از میزانی باشد که با اهمیت انسان دوستانه مشکلی که باید توسط آزمایش حل شود تعیین می شود.
        \item[۷] باید آماده‌سازی مناسب و امکانات کافی برای محافظت از آزمودنی آزمایشی در برابر احتمالات دوردست آسیب، ناتوانی یا مرگ فراهم شود.
        \item[۸] آزمایش فقط باید توسط افراد واجد شرایط علمی انجام شود.
        بالاترین درجه مهارت و مراقبت باید در تمام مراحل آزمایش کسانی که آزمایش را انجام می دهند یا درگیر آن هستند، لازم باشد.
        \item[۹] در طول آزمایش، آزمودنی انسانی باید آزاد باشد که آزمایش را به پایان برساند، اگر به وضعیت جسمی یا روانی رسیده باشد که ادامه آزمایش به نظر او غیرممکن است.
        \item[۱۰] در طول آزمایش، دانشمند مسئول باید آمادگی داشته باشد که آزمایش را در هر مرحله خاتمه دهد، اگر احتمالاً دلایلی برای باور داشته باشد، به اعمال حسن‌نیت، مهارت برتر و قضاوت دقیق که از او مستلزم ادامه است.
        این آزمایش احتمالاً منجر به آسیب، ناتوانی یا مرگ آزمودنی می شود.
    \end{description}
\end{tcolorbox}

