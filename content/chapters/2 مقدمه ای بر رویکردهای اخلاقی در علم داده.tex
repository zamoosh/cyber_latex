%! Author = zamoosh
%! Date = 6/4/23


\chapter{مقدمه ای بر رویکردهای اخلاقی در علم داده}
\label{ch:مقدمه ای بر رویکردهای اخلاقی در علم داده}

\begin{quote}
    دانش علم فیزیک مرا به خاطر ناآگاهی از اخلاق، تسلی نمی‌دهد، اما علم اخلاق همیشه مرا به خاطر ناآگاهی از علم فیزیکی تسلی می‌دهد.
    (منظور نویسنده، تأکید مهم بودن علم اخلاق و ارزش‌های انسانی است)
    \begin{flushleft}
        \textenglish{\textbf{Blaise Pascal} \LR{1624}-\LR{1624}}
    \end{flushleft}
\end{quote}

\phantomsection
\section*{مقدمه}
\addcontentsline{toc}{section}{مقدمه}{\protect\numberline{}}
فناوری‌های یادگیری ماشین، در حال نفوذ به زندگی مردم عادی در سراسر جهان هستند.\     كاربران اين فناوری‌ها، خواسته یا ناخواسته در زندگی اصولی دارند که رویکردهای آن، در میان فلفسه‌های غربی ارائه نشده.
بنابراین، ما چندین رویکرد اخلاقی غیر غربی را در کتاب آورده‌ایم.
این‌ها برای طراحان ارزش دانستن دارد، هم برای اینکه بتوانند کاوش اخلاقی خود را عمیق‌تر کنند و هم به این ترتیب که بتوانند بهتر درک کنند که چگونه فن آوری‌هایشان تفسیر، اتخاذ، استفاده و تنظیم می‌شود.
ما خوش‌شانس بوده‌ایم که تفسیرهایی از دانشمندان برجسته در زمینه‌های اخلاق دئونتولوژیک، اخلاق نتیجه‌گرا (فایده گرا)، و اخلاق فضیلت و فطری، و همچنین از اخلاق اوبونتو، اخلاق بودایی، اخلاق یهودی، و اخلاق بومی و ذاتی دریافت کرده‌ایم.
ما امیدواریم که این به خواننده دید وسیع‌تری بدهد تا درباره‌ی فناوری‌های یادگیری ماشین از دیدگاه‌های مختلف فکر کند و بفهمد که چگونه آن‌ها توسط جوامع سراسر جهان پذیرفته می‌شوند و چگونه عمل می‌کنند.
هر یک از این رویکردهای اخلاقی در زیر به اختصار آورده شده است.

\phantomsection
\section*{رفتار نتیجه گرایی و فایده‌گرایی}
\addcontentsline{toc}{section}{رفتار نتیجه گرایی و فایده‌گرایی}{\protect\numberline{}}
\begin{quote}
    توسط پیتر سینگر و ییپ فای تسه
\end{quote}

نتیجه‌گرایی خانواده‌ای از نظریه‌ها است که بر این عقیده هستند که درست یا نادرست بودن یک عمل بستگی به پیامدهای آن دارد یا به عبارت دیگر، وضعیتی که اعمال باعث ایجاد آن می‌شود.
فایده‌گرایی، در شکل کلاسیک خود، نظریه نتیجه گرایی است که منحصراً بر درد و لذت، یا شادی و بدبختی، به عنوان تنها پیامدهای اخلاقی مرتبط برای تعیین چگونگی ارزیابی پیامدهای اعمال تمرکز می کند.
در اینجا تأکید بر این نکته حائز اهمیت است که فایده‌گرایی تنها در مورد ارزیابی درستی یا نادرستی اعمال نیست، بلکه در مورد ارزیابی خوب و بد حالت‌های امور است، که بی طرفانه در نظر گرفته می‌شوند.
به طور خاص، فایده‌گرایان معتقدند که همه‌ی موجودات ذی‌شعور (آن‌هایی که می‌توانند درد و لذت را تجربه کنند) باید در نظر گرفته‌شوند و به علایق مشابه آن‌ها باید وزن مشابهی داده شود.
در کنار هم، فایده‌گرایی، این دیدگاه است که یک عمل نه تنها باید منفعت برساند، بلکه از نظر اخلاقی نیز لازم است که بیشترین مازاد خالص ممکن را از شادی نسبت به بدبختی (یا لذت بر درد) به همراه داشته باشد.
و هر عملی که بر خلاف این اصل باشد، ممنوع و غیرمجاز است.

\subsection*{اعتراضات رایج به سودگرایی}
\addcontentsline{toc}{subsection}{اعتراضات رایج به سودگرایی}{\protect\numberline{}}
یک اعتراض رایج به سودگرایی این است که ما را به انجام اعمال آشکاراً غیراخلاقی هدایت می‌کند!
«داستایوفسکی» در «برادران کارامازوف»، «ایوان» را به چالش می کشدکه یک نوزاد را تا سرحد مرگ شکنجه کند تا برای همه‌ی بشریت خوشبختی بیاورد.
چالش «ایوان» به یک اعتراض معروف به سودگرایی تبدیل شده است.
بیان ساختار اعتراض «داستایوفسکی» به طور رسمی این موضوع را بهتر نشان می دهد:
\\\\
\textbf{فرض 1.}
اگر فایده‌گرایی درست بود، به درستی به ما می گفت که کدام اعمال درست و کدام نادرست است.
\\\\
\textbf{فرض 2.}\     فایده‌گرایی به ما می گوید که اگر شکنجه‌ی یک کودک بی گناه تا حد مرگ عواقب بهتری نسبت به هر عمل دیگری به همراه داشته باشد، آنگاه شکنجه یک کودک بی گناه تا حد مرگ کار درستی خواهد بود.
\\\\
\textbf{فرض 3.}
شکنجه یک کودک بی گناه تا حد مرگ همیشه اشتباه است.
نتیجه: فایده‌گرایی نادرست است.
\\\\
بسیاری از ایرادات به فایده‌گرایی نیز به همین ترتیب مطرح می شوند: یک جراح به این فکر می‌کند که آیا مخفیانه اطمینان حاصل کند که یک عمل شکست می خورد؛ تا بیمار بمیرد و سپس از اعضای بدن او برای نجات جان چهار بیمار در انتظار اهدای اعضای ضروری استفاده شود.
چنین نمونه‌هایی منعکس کننده‌ی دانش ما از نحوه عملکرد جهان نیستند.
«ایوان» توضیح نداد که چگونه شکنجه‌ی کودک باعث شادی پایدار برای دیگران می‌شود.
مثال پیوند عضو در نظر نمی‌گیرد که اگر کاری که جراح انجام داده مشخص شود، ممکن است منجر به عواقبی شود که بسیار بیشتر از مزایای مورد نظر است (ممکن است افراد نسبت به پزشکان بی‌اعتماد شوند).
چگونه جراح می تواند کاملاً مطمئن باشد که او گرفتار نخواهد شد؟ این فرض که شکنجه یک کودک بی گناه همیشه اشتباه است، متکی به ذات و فطرت انسانی دارد.
بنابراین وقتی با نمونه‌های عجیب و خیالی سروکار داریم، فرض 3 مشکوک است و نمی‌توان به آن به عنوان مبنایی برای رد فایده‌گرایی اعتماد کرد.

ایراد اصلی دیگر این است که اندازه‌گیری درد و لذت، یا شادی و غم است.\     سودگرایان سه پاسخ اصلی به این اعتراض دارند.
اولاً، این مشکلی محدود به فایده‌گرایی نیست.
هر نظریه‌ی اخلاقی‌ای که مقداری به رفاه اهمیت می‌دهد از دشواری اندازه‌گیری رفاه افرادی که تحت تأثیر اعمال هستند نیز رنج می برد؛ و البته نظریه‌ی اخلاقی‌ای که تمام این ملاحظات رفاهی را نادیده می‌گیرد بسیار غیرقابل‌قبول خواهد بود.

ثانیاً، اگرچه اندازه‌گیری دقیق درد و لذت دشوار است، ترجیحات افراد و تا حدی حیوانات را می توان مشاهده، آزمایش و رتبه بندی کرد تا اولویت‌های آن‌ها آشکار مشخص.\     در برخی از مطالعات، روانشناسان با پرداخت هزینه به آزمایش‌شوندگان، سطوح خاصی از درد یا تحمل را در آن‌ها می‌سنجند.
این موارد، اگرچه آن چیزی نیست که فایده‌گرایان کلاسیک آن را خیر می‌دانند، با این وجود، معیارهای مفیدی هستند که به ما ایده‌ای درباره‌ی درد و لذت می‌دهند.
مدل دیگری که از موارد آشکار استفاده می‌کند، سال زندگی تعدیل‌شده با کیفیت یا \textenglish{\textbf{(QALY)}}، حول این ایده است که یک سال زندگی با عملکرد یا سلامت مختل، به اندازه یک سال در سلامت عادی، خوب نیست.
برای مثال، محققان از مردم می‌خواهند که خود را با آسیب‌های مختلف در سلامت تصور کنند (گاهی اوقات خود درد)، و سپس از آن‌ها می‌پرسند که حاضرید چند سال از زندگی خود را رها کنید تا این اختلال درمان شود؟ این روش اکنون در سطح جهانی توسط اقتصاددانان سلامت، محققان پزشکی و سیاست‌گذاران استفاده می‌شود.
در نهایت، در اکثر موارد، عمل درست حتی بدون اندازه‌گیری واضح است.
به عنوان مثال، پزشکی که ترتیب درد بیماران را در اولویت قرار می‌دهد، می تواند به وضوح ببیند که یک بیمار سوختگی، شدیدتر از فردی که از سرماخوردگی رنج می برد، درد دارد و در معرض خسر مرگ بسیار بالاتری است؛ بنابراین، باید بیمار سوختگی را در اولیت قرار داد.
یا مثلا اگر فردی از شما بپرسد که نزدیک‌ترین رستوران گیاه‌خواران کجاست؟ شما به احتمال خیلی زیاد با ارائه‌ی اطلاعات درست، او را راهنمایی می‌کنید، تا اینکه اصلا جواب ندهید یا پاسخ اشتباه بدهید!

اگرچه مواردی هم وجود دارند که پس از تجزیه و تحلیل هم شفاف نیستند؛ ولی با این جود می‌توان تصمیمات معقولی گرفت.\     نکته‌ی مهمی که در اینجا باید مورد توجه قرار گیرد، این است که نه تنها می توان بخش قابل توجهی از تصمیمات تحت فایده‌گرایی را بدون اندازه‌گیری لذت و درد اتخاذ کرد، بلکه آنچه در این دنیا در خطر است نیز معمولاً می‌تواند بدون اندازه‌گیری مستقیم درد و لذت تعیین شود.
فقر جهانی (که باعث گرسنگی، تشنگی، بیماری‌ها و ... می‌شوند)، کشاورزی کارخانه‌ای و بیماری‌های همه‌گیر نمونه‌های مناسبی از مسائلی هستند که بدون شک، رنج عظیمی را برای تعداد زیادی از افراد به بار می‌آورند.


\subsection*{توصیه‌هایی برای به کارگیری صحیح اصول سودمندی}
\addcontentsline{toc}{subsection}{توصیه‌هایی برای به کارگیری صحیح اصول سودمندی}{\protect\numberline{}}


\subsubsection*{گسترده تر و طولانی تر فکر کنید}
\addcontentsline{toc}{subsubsection}{گسترده تر و طولانی تر فکر کنید}{\protect\numberline{}}
ما با «جان استوارت میل»، یک فایده‌گرای اولیه، موافقیم که باید «مفید بودن را به‌عنوان اصل نهایی در همه مسائل اخلاقی در نظر بگیریم.\     اما باید در فراگیرترین معنای آن فایده باشد».
منظور ما از "فراگیرترین" این است که همه پیامدهای مرتبط، صرف نظر از زمان، فاصله فیزیکی، خویشاوندی و سایر ویژگی‌های اخلاقی نامربوط مانند جنسیت، نژاد، و عضویت در گونه باید در نظر گرفته شوند.

مسلماً، زمان یکی از بحث برانگیزترین آن‌هاست که از نظر اخلاقی نامربوط اعلام می‌شود.
تخفیف زمان اغلب در زمینه‌های اقتصاد و یادگیری ماشین آموزش داده‌می‌شود و به کار می‌رود، ولی تصورات آن‌ها در مورد ترجیحات زمانی با ایده‌های فایده‌گرایی متفاوت است.
در اقتصاد، کاهش زمان، برای دریافت لذت و خوشی در زمان کمتر مد نظر است؛ یعنی ما مایل هستیم که لذت و خوشی را در زمان کمتری به دست بیاوریم تا اینکه بخواهیم برای آن صبر کنیم.
در یادگیری ماشین به ویژه یادگیری تقویتی، "ضریب تخفیف" $\symup{(𝛾)}$، متغیری است که تعیین می‌کند که عامل، تمایل به اهداف و پاداش‌های زودهنگام دارد یا دیرهنگام (اهمیت را برای پاداش‌های فوری یا آینده تعیین می‌کند).\     اگر مقدار $\symup{(𝛾)}$ نزدیک به 1 باشد، عامل به پاداش‌های آینده، بیشتر اهمیت می‌دهد و در نتیجه تمایل دارد تا مسیری را که باعث رسیدن به هدف در آینده می‌شود، دنبال کند.
به عبارتی، عامل تمایل دارد پاداش‌های آینده را بیشتر به صورت بلندمدت مد نظر قرار دهد.
از سوی دیگر، اگر مقدار $\symup{(𝛾)}$ نزدیک به 0 باشد، عامل بیشتر روی پاداش‌های فوری تمرکز می‌کند و تمایل دارد که از پاداش‌های فوری بهره‌برداری کند.
به عبارتی، عامل در تصمیم‌گیری خود بیشتر به جوانب کوتاه‌مدت توجه می‌کند و پاداش‌های آینده، اهمیت نمی‌دهد.
مثلا می‌توانیم بگوئیم به دلیل اینکه در آینده فلان بازار هدف وجود نخواهد داشت، $\symup{(𝛾)}$ را نزدیک به 0 د نظر می‌گیریم تا در کوتاه مدت، به نتیجه‌ی دلخواه برسیم، عکس این قضیه هم صادق است.
به عنوان مثال، شکنجه در 100 سال به همان اندازه بد است که شکنجه‌ای اکنون به همان اندازه درد داشته‌باشد، اما اگر قطعیت کمتری داشته باشد (یعنی ممکن باشد که شکنجه انجام نشود)، ممکن است به همین دلیل آن را کاهش دهیم (یعنی شکنجه‌ی چیزی را می‌پذیریم که مارا شکنجه نکند یا آن موردی که قطعیت کمتری دارد).

بیایید سعی کنیم این اصول را در هوش مصنوعی و علم داده اعمال کنیم.
برای مثال، در تصمیم‌گیری برای راه‌اندازی یک محصول، نه تنها باید تاثیری که ممکن است بر روی کاربران آن داشته باشد، بلکه باید در نظر داشت که چگونه جامعه وسیع‌تر افراد (در برخی موارد، حتی حیوانات) چه در کوتاه مدت و چه در بلند مدت ممکن است تحت تاثیر قرار گیرند.
سؤالاتی از این قبیل باید پرسیده شود: آیا این محصول سوگیری‌ها، فرهنگ، ایدئولوژی‌ها، فضیلت‌ها یا سایر ارزش‌ها را در جامعه جذب و در نتیجه آن را تقویت می کند؟ آیا این محصول، یک صنعت بسیار ارزشمند را از بین می برد یا باعث به تاخیر انداختن یا جلوگیری از حذف یک صنعت غیراخلاقی می‌شود؟


%    refrence\textsuperscript{\ref{note1}}
%\phantomsection
%    \section*{Notes}
%    \label{sec:notes}
%    \begin{enumerate}
%        \item \label{note1}
%        \item \label{note2} ali ali ali
%        \item This is another note.
%    \end{enumerate}


\subsubsection*{از ارزش‌های مورد انتظار برای تصمیم گیری استفاده کنید}
\addcontentsline{toc}{subsubsection}{از ارزش‌های مورد انتظار برای تصمیم گیری استفاده کنید}{\protect\numberline{}}
استفاده از تئوری ارزش مورد انتظار در تصمیم‌گیری، در تئوری تصمیم گیری، اقتصاد و علم داده، اساسی است (یعنی قبل از تصمیم‌گیری بسنجیم ببینم که دنبال چه چیزی هستیم و بر اساس آن تصمیم‌گیری بکنیم).
اما باید در مورد نظریه‌های اخلاقی، به‌ویژه به حداکثر رساندن فاکتورهای اخلاقی مانند فایده‌گرایی نیز اعمال شود.
مثال جراح در بخش قبل نشان می‌دهد که چرا سناریوهایی با ریسک بالا و کم احتمال، اهمیت دارند.
مهم نیست که جراح چقدر با دقت سعی کرد عمل او را مخفی نگه دارد، او نتوانست به طور منطقی به این نتیجه برسد که احتمال افشای راز صفر است.
با توجه به اثرات مشخص کشف شدن راز (اگر کشف می‌شد، مردم نسبت به پزشکان اعتمادشان را از دست می‌دادند)، جراح باید به این نتیجه برسد که انجام چنین عملی اشتباه است.

در حالی که محاسبه ارزش مورد انتظار اغلب ساده است (انجام آن عمل، چندین انسان را نجات می‌داد)، به دلیل سوگیری‌های شناختی انسان (مثلا اینکه شما بیمار من رو به خاطر اهدای عضو، به عمد به قتل رساندید!)، اغلب به درستی استفاده نمی‌شود یا حتی اصلاً اعمال نمی‌شود.
«غفلت احتمالی» یک سوگیری شناختی است که افراد نسبت به «عدم قطعیت‌ها» نشان می‌دهند، به‌ویژه «احتمالات کوچک»، که تمایل دارند یا به طور کامل از آنها غفلت کنند، یا تا حد زیادی (اغراق) آن را بزرگ کنند.
یک مطالعه با دریافت اینکه مردم برای کاهش خطرات "رویدادهای نادر و پر تاثیر" یا ارزش خیلی زیاد یا بسیار پایین قائل هستند؛ (غفلت احتمالی) را تائید کرد.
ما نیازی به جستجوی شواهدی مبنی بر غفلت جمعی از «رویدادهای نادر و پرتأثیر» نداریم!
اگر قانون اجباری بستن کمربند در خودرو برداشته‌شود، به نظر شما چند نفر حاضرند تا کمربندشان ببیندند؟ (این خود نشان دهنده‌ی این است که مردم از سوگیری شناختی غفلت احتمالی استفاده می‌کنند!) این قضیه اصلا هم جالب نیست!
زیرا «رویدادهایی با احتمال کم و تأثیر زیاد» اغلب دارای ارزش‌های مورد انتظار بزرگ، اعم از منفی یا مثبت هستند، این دام در تفکر انسان نگران‌کننده است!
این نشان می‌دهد که انسان اغلب به «ارزش‌های مورد انتظار» حتی فکر هم نمی‌کند!
چه برسد که بخواهد آن را هنگام تصمیم‌گیری به کار ببرد!

سوگیری دیگری که ممکن است بر توانایی افراد در برآورد مقادیر مورد انتظار تأثیر بگذارد، «غفلت از محدوده» است.\     مطالعات نشان داده است که افراد ارزش‌گذاری خود را در تناسب با مقیاس یک مسئله تنظیم نمی‌کنند.
به عنوان مثال، یک مطالعه از سه گروه از افراد در مورد تمایل آن ها به پرداخت هزینه برای نجات 2000 یا 20000 یا 200000 پرنده از غرق شدن در استخرهای نفتی بدون سرپوش پرسیده شد.
میانگین‌های مربوطه 80، 78 و 88 دلار و میانگین پاسخ‌ها همگی 25 دلار بود.
اگر ارزش‌گذاری افراد از برخی نتایج به‌درستی مقیاس‌پذیر نباشد، ارزش‌های مورد انتظار نیز نخواهد بود (یعنی اینجا باید هرکس با توجه به دارایی خود مبلغی را اعلام می‌کرد، ولی همه‌ی آن‌ها پاسخی نزدیک به 25 دلار داده‌بودند).


\subsubsection*{در انتخاب پروژه های خیریه، پروژه‌های (موارد) موثر را انتخاب کنید}
\addcontentsline{toc}{subsubsection}{در انتخاب پروژه های خیریه، پروژه‌های (موارد) موثر را انتخاب کنید}{\protect\numberline{}}
از آنجایی که مردم معمولاً به جای تحقیق در مورد اثربخشی خیریه، بر اساس انگیزه و احساسات به خیریه می‌پردازند، اغلب از خیریه‌ها و اهداف بی‌اثر حمایت می‌کنند.\     ولی در عوض چیزاهایی نذیر: نوع دوستی مؤثر، یک جنبش جهانی اخیر، بر اهمیت رفتار نوع دوستانه مؤثر، چه در قالب کمک های مالی و چه در قالب زمان مهم هستند!

چه خوب است که همین اصل (سراغ کارهایی برویم که اثربخشی بالا دارند) را در هوش مصنوعی پیاده‌سازی کنیم و به اهداف مهم‌تر، اولویت بالاتری بدهیم.





\backmatter