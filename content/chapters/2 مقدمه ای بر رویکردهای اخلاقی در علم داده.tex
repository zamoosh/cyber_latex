%! Author = zamoosh
%! Date = 6/4/23


\chapter{مقدمه ای بر رویکردهای اخلاقی در علم داده}
\label{ch:مقدمه ای بر رویکردهای اخلاقی در علم داده}
\phantomsection

\begin{quote}
    دانش علم فیزیک مرا به خاطر ناآگاهی از اخلاق، تسلی نمی‌دهد، اما علم اخلاق همیشه مرا به خاطر ناآگاهی از علم فیزیکی تسلی می‌دهد.
    (منظور نویسنده، تأکید مهم بودن علم اخلاق و ارزش‌های انسانی است)
    \begin{flushleft}
        \textenglish{\textbf{Blaise Pascal} \LR{1624}-\LR{1624}}
    \end{flushleft}
\end{quote}

\phantomsection
\section*{مقدمه}
\label{sec:مقدمه}
\addcontentsline{toc}{section}{مقدمه}{\protect\numberline{}}
فناوری‌های یادگیری ماشین، در حال نفوذ به زندگی مردم عادی در سراسر جهان هستند.\     كاربران اين فناوری‌ها، خواسته یا ناخواسته در زندگی اصولی دارند که رویکردهای آن، در میان فلفسه‌های غربی ارائه نشده.
بنابراین، ما چندین رویکرد اخلاقی غیر غربی را در کتاب آورده‌ایم.
این‌ها برای طراحان ارزش دانستن دارد، هم برای اینکه بتوانند کاوش اخلاقی خود را عمیق‌تر کنند و هم به این ترتیب که بتوانند بهتر درک کنند که چگونه فن آوری‌هایشان تفسیر، اتخاذ، استفاده و تنظیم می‌شود.
ما خوش‌شانس بوده‌ایم که تفسیرهایی از دانشمندان برجسته در زمینه‌های اخلاق دئونتولوژیک، اخلاق نتیجه‌گرا (فایده گرا)، و اخلاق فضیلت و فطری، و همچنین از اخلاق اوبونتو، اخلاق بودایی، اخلاق یهودی، و اخلاق بومی و ذاتی دریافت کرده‌ایم.
ما امیدواریم که این به خواننده دید وسیع‌تری بدهد تا درباره‌ی فناوری‌های یادگیری ماشین از دیدگاه‌های مختلف فکر کند و بفهمد که چگونه آن‌ها توسط جوامع سراسر جهان پذیرفته می‌شوند و چگونه عمل می‌کنند.
هر یک از این رویکردهای اخلاقی در زیر به اختصار آورده شده است.

\phantomsection

\section*{رفتار نتیجه گرایی و فایده‌گرایی}
\label{sec:رفتار نتیجه گرایی و فایده‌گرایی}
\addcontentsline{toc}{section}{رفتار نتیجه گرایی و فایده‌گرایی}{\protect\numberline{}}
\begin{quote}
    توسط پیتر سینگر و ییپ فای تسه
\end{quote}

نتیجه‌گرایی خانواده‌ای از نظریه‌ها است که بر این عقیده هستند که درست یا نادرست بودن یک عمل بستگی به پیامدهای آن دارد یا به عبارت دیگر، وضعیتی که اعمال باعث ایجاد آن می‌شود.
فایده‌گرایی، در شکل کلاسیک خود، نظریه نتیجه گرایی است که منحصراً بر درد و لذت، یا شادی و بدبختی، به عنوان تنها پیامدهای اخلاقی مرتبط برای تعیین چگونگی ارزیابی پیامدهای اعمال تمرکز می‌کند.
در اینجا تأکید بر این نکته حائز اهمیت است که فایده‌گرایی تنها در مورد ارزیابی درستی یا نادرستی اعمال نیست، بلکه در مورد ارزیابی خوب و بد حالت‌های امور است، که بی طرفانه در نظر گرفته می‌شوند.
به طور خاص، فایده‌گرایان معتقدند که همه‌ی موجودات ذی‌شعور (آن‌هایی که می‌توانند درد و لذت را تجربه کنند) باید در نظر گرفته‌شوند و به علایق مشابه آن‌ها باید وزن مشابهی داده شود.
در کنار هم، فایده‌گرایی، این دیدگاه است که یک عمل نه تنها باید منفعت برساند، بلکه از نظر اخلاقی نیز لازم است که بیشترین مازاد خالص ممکن را از شادی نسبت به بدبختی (یا لذت بر درد) به همراه داشته باشد.
و هر عملی که بر خلاف این اصل باشد، ممنوع و غیرمجاز است.

\newpage

\phantomsection
\subsection*{اعتراضات رایج به سودگرایی}
\label{subsec:اعتراضات رایج به سودگرایی}
\addcontentsline{toc}{subsection}{اعتراضات رایج به سودگرایی}{\protect\numberline{}}
یک اعتراض رایج به سودگرایی این است که ما را به انجام اعمال آشکاراً غیراخلاقی هدایت می‌کند!
«داستایوفسکی» در «برادران کارامازوف»، «ایوان» را به چالش می کشدکه یک نوزاد را تا سرحد مرگ شکنجه کند تا برای همه‌ی بشریت خوشبختی بیاورد.
چالش «ایوان» به یک اعتراض معروف به سودگرایی تبدیل شده است.
بیان ساختار اعتراض «داستایوفسکی» به طور رسمی این موضوع را بهتر نشان می‌دهد:
\\\\
\textbf{فرض ۱.}
اگر فایده‌گرایی درست بود، به درستی به ما می گفت که کدام اعمال درست و کدام نادرست است.
\\\\
\textbf{فرض ۲.}\     فایده‌گرایی به ما می گوید که اگر شکنجه‌ی یک کودک بی گناه تا حد مرگ عواقب بهتری نسبت به هر عمل دیگری به همراه داشته باشد، آنگاه شکنجه یک کودک بی گناه تا حد مرگ کار درستی خواهد بود.
\\\\
\textbf{فرض ۳.}
شکنجه یک کودک بی گناه تا حد مرگ همیشه اشتباه است.
نتیجه: فایده‌گرایی نادرست است.
\\\\
بسیاری از ایرادات به فایده‌گرایی نیز به همین ترتیب مطرح می شوند: یک جراح به این فکر می‌کند که آیا مخفیانه اطمینان حاصل کند که یک عمل شکست می خورد؛ تا بیمار بمیرد و سپس از اعضای بدن او برای نجات جان چهار بیمار در انتظار اهدای اعضای ضروری استفاده شود.
چنین نمونه‌هایی منعکس کننده‌ی دانش ما از نحوه عملکرد جهان نیستند.
«ایوان» توضیح نداد که چگونه شکنجه‌ی کودک باعث شادی پایدار برای دیگران می‌شود.
مثال پیوند عضو در نظر نمی‌گیرد که اگر کاری که جراح انجام داده مشخص شود، ممکن است منجر به عواقبی شود که بسیار بیشتر از مزایای مورد نظر است (ممکن است افراد نسبت به پزشکان بی‌اعتماد شوند).
چگونه جراح می‌تواند کاملاً مطمئن باشد که او گرفتار نخواهد شد؟ این فرض که شکنجه یک کودک بی گناه همیشه اشتباه است، متکی به ذات و فطرت انسانی دارد.
بنابراین وقتی با نمونه‌های عجیب و خیالی سروکار داریم، فرض 3 مشکوک است و نمی‌توان به آن به عنوان مبنایی برای رد فایده‌گرایی اعتماد کرد.

ایراد اصلی دیگر این است که اندازه‌گیری درد و لذت، یا شادی و غم است.\     سودگرایان سه پاسخ اصلی به این اعتراض دارند.
اولاً، این مشکلی محدود به فایده‌گرایی نیست.
هر نظریه‌ی اخلاقی‌ای که مقداری به رفاه اهمیت می‌دهد از دشواری اندازه‌گیری رفاه افرادی که تحت تأثیر اعمال هستند نیز رنج می برد؛ و البته نظریه‌ی اخلاقی‌ای که تمام این ملاحظات رفاهی را نادیده می‌گیرد بسیار غیرقابل‌قبول خواهد بود.

ثانیاً، اگرچه اندازه‌گیری دقیق درد و لذت دشوار است، ترجیحات افراد و تا حدی حیوانات را می توان مشاهده، آزمایش و رتبه بندی کرد تا اولویت‌های آن‌ها آشکار مشخص.\     در برخی از مطالعات، روانشناسان با پرداخت هزینه به آزمایش‌شوندگان، سطوح خاصی از درد یا تحمل را در آن‌ها می‌سنجند.
این موارد، اگرچه آن چیزی نیست که فایده‌گرایان کلاسیک آن را خیر می‌دانند، با این وجود، معیارهای مفیدی هستند که به ما ایده‌ای درباره‌ی درد و لذت می‌دهند.
مدل دیگری که از موارد آشکار استفاده می‌کند، سال زندگی تعدیل‌شده با کیفیت یا \textenglish{\textbf{(QALY)}}، حول این ایده است که یک سال زندگی با عملکرد یا سلامت مختل، به اندازه یک سال در سلامت عادی، خوب نیست.
برای مثال، محققان از مردم می‌خواهند که خود را با آسیب‌های مختلف در سلامت تصور کنند (گاهی اوقات خود درد)، و سپس از آن‌ها می‌پرسند که حاضرید چند سال از زندگی خود را رها کنید تا این اختلال درمان شود؟ این روش اکنون در سطح جهانی توسط اقتصاددانان سلامت، محققان پزشکی و سیاست‌گذاران استفاده می‌شود.
در نهایت، در اکثر موارد، عمل درست حتی بدون اندازه‌گیری واضح است.
به عنوان مثال، پزشکی که ترتیب درد بیماران را در اولویت قرار می‌دهد، می‌تواند به وضوح ببیند که یک بیمار سوختگی، شدیدتر از فردی که از سرماخوردگی رنج می برد، درد دارد و در معرض خسر مرگ بسیار بالاتری است؛ بنابراین، باید بیمار سوختگی را در اولیت قرار داد.
یا مثلا اگر فردی از شما بپرسد که نزدیک‌ترین رستوران گیاه‌خواران کجاست؟ شما به احتمال خیلی زیاد با ارائه‌ی اطلاعات درست، او را راهنمایی می‌کنید، تا اینکه اصلا جواب ندهید یا پاسخ اشتباه بدهید!

اگرچه مواردی هم وجود دارند که پس از تجزیه و تحلیل هم شفاف نیستند؛ ولی با این جود می‌توان تصمیمات معقولی گرفت.\     نکته‌ی مهمی که در اینجا باید مورد توجه قرار گیرد، این است که نه تنها می توان بخش قابل توجهی از تصمیمات تحت فایده‌گرایی را بدون اندازه‌گیری لذت و درد اتخاذ کرد، بلکه آنچه در این دنیا در خطر است نیز معمولاً می‌تواند بدون اندازه‌گیری مستقیم درد و لذت تعیین شود.
فقر جهانی (که باعث گرسنگی، تشنگی، بیماری‌ها و ... می‌شوند)، کشاورزی کارخانه‌ای و بیماری‌های همه‌گیر نمونه‌های مناسبی از مسائلی هستند که بدون شک، رنج عظیمی را برای تعداد زیادی از افراد به بار می‌آورند.

\phantomsection
\subsection*{توصیه‌هایی برای به کارگیری صحیح اصول سودمندی}
\label{subsec:توصیه‌هایی برای به کارگیری صحیح اصول سودمندی}
\addcontentsline{toc}{subsection}{توصیه‌هایی برای به کارگیری صحیح اصول سودمندی}{\protect\numberline{}}


\phantomsection
\subsubsection*{گسترده تر و طولانی تر فکر کنید}
\label{subsubsec:گسترده تر و طولانی تر فکر کنید}
\addcontentsline{toc}{subsubsection}{گسترده تر و طولانی تر فکر کنید}{\protect\numberline{}}
ما با «جان استوارت میل»، یک فایده‌گرای اولیه، موافقیم که باید «مفید بودن را به‌عنوان اصل نهایی در همه مسائل اخلاقی در نظر بگیریم.\     اما باید در فراگیرترین معنای آن فایده باشد».
منظور ما از "فراگیرترین" این است که همه پیامدهای مرتبط، صرف نظر از زمان، فاصله فیزیکی، خویشاوندی و سایر ویژگی‌های اخلاقی نامربوط مانند جنسیت، نژاد، و عضویت در گونه باید در نظر گرفته شوند.

مسلماً، زمان یکی از بحث برانگیزترین آن‌هاست که از نظر اخلاقی نامربوط اعلام می‌شود.
تخفیف زمان اغلب در زمینه‌های اقتصاد و یادگیری ماشین آموزش داده‌می‌شود و به کار می‌رود، ولی تصورات آن‌ها در مورد ترجیحات زمانی با ایده‌های فایده‌گرایی متفاوت است.
در اقتصاد، کاهش زمان، برای دریافت لذت و خوشی در زمان کمتر مد نظر است؛ یعنی ما مایل هستیم که لذت و خوشی را در زمان کمتری به دست بیاوریم تا اینکه بخواهیم برای آن صبر کنیم.
در یادگیری ماشین به ویژه یادگیری تقویتی، "ضریب تخفیف" $\symup{(𝛾)}$، متغیری است که تعیین می‌کند که عامل، تمایل به اهداف و پاداش‌های زودهنگام دارد یا دیرهنگام (اهمیت را برای پاداش‌های فوری یا آینده تعیین می‌کند).\     اگر مقدار $\symup{(𝛾)}$ نزدیک به 1 باشد، عامل به پاداش‌های آینده، بیشتر اهمیت می‌دهد و در نتیجه تمایل دارد تا مسیری را که باعث رسیدن به هدف در آینده می‌شود، دنبال کند.
به عبارتی، عامل تمایل دارد پاداش‌های آینده را بیشتر به صورت بلندمدت مد نظر قرار دهد.
از سوی دیگر، اگر مقدار $\symup{(𝛾)}$ نزدیک به 0 باشد، عامل بیشتر روی پاداش‌های فوری تمرکز می‌کند و تمایل دارد که از پاداش‌های فوری بهره‌برداری کند.
به عبارتی، عامل در تصمیم‌گیری خود بیشتر به جوانب کوتاه‌مدت توجه می‌کند و پاداش‌های آینده، اهمیت نمی‌دهد.
مثلا می‌توانیم بگوئیم به دلیل اینکه در آینده فلان بازار هدف وجود نخواهد داشت، $\symup{(𝛾)}$ را نزدیک به 0 د نظر می‌گیریم تا در کوتاه مدت، به نتیجه‌ی دلخواه برسیم، عکس این قضیه هم صادق است.
به عنوان مثال، شکنجه در 100 سال به همان اندازه بد است که شکنجه‌ای اکنون به همان اندازه درد داشته‌باشد، اما اگر قطعیت کمتری داشته باشد (یعنی ممکن باشد که شکنجه انجام نشود)، ممکن است به همین دلیل آن را کاهش دهیم (یعنی شکنجه‌ی چیزی را می‌پذیریم که مارا شکنجه نکند یا آن موردی که قطعیت کمتری دارد).

بیایید سعی کنیم این اصول را در هوش مصنوعی و علم داده اعمال کنیم.
برای مثال، در تصمیم‌گیری برای راه‌اندازی یک محصول، نه تنها باید تاثیری که ممکن است بر روی کاربران آن داشته باشد، بلکه باید در نظر داشت که چگونه جامعه وسیع‌تر افراد (در برخی موارد، حتی حیوانات) چه در کوتاه مدت و چه در بلند مدت ممکن است تحت تاثیر قرار گیرند.
سؤالاتی از این قبیل باید پرسیده شود: آیا این محصول سوگیری‌ها، فرهنگ، ایدئولوژی‌ها، فضیلت‌ها یا سایر ارزش‌ها را در جامعه جذب و در نتیجه آن را تقویت می‌کند؟ آیا این محصول، یک صنعت بسیار ارزشمند را از بین می برد یا باعث به تاخیر انداختن یا جلوگیری از حذف یک صنعت غیراخلاقی می‌شود؟


%    refrence\textsuperscript{\ref{note1}}
%\phantomsection
%    \section*{Notes}
%    \label{sec:notes}
%    \begin{enumerate}
%        \item \label{note1}
%        \item \label{note2} ali ali ali
%        \item This is another note.
%    \end{enumerate}

\phantomsection
\subsubsection*{از ارزش‌های مورد انتظار برای تصمیم گیری استفاده کنید}
\label{subsubsec:از ارزش‌های مورد انتظار برای تصمیم گیری استفاده کنید}
\addcontentsline{toc}{subsubsection}{از ارزش‌های مورد انتظار برای تصمیم گیری استفاده کنید}{\protect\numberline{}}
استفاده از تئوری ارزش مورد انتظار در تصمیم‌گیری، در تئوری تصمیم گیری، اقتصاد و علم داده، اساسی است (یعنی قبل از تصمیم‌گیری بسنجیم ببینم که دنبال چه چیزی هستیم و بر اساس آن تصمیم‌گیری بکنیم).
اما باید در مورد نظریه‌های اخلاقی، به‌ویژه به حداکثر رساندن فاکتورهای اخلاقی مانند فایده‌گرایی نیز اعمال شود.
مثال جراح در بخش قبل نشان می‌دهد که چرا سناریوهایی با ریسک بالا و کم احتمال، اهمیت دارند.
مهم نیست که جراح چقدر با دقت سعی کرد عمل او را مخفی نگه دارد، او نتوانست به طور منطقی به این نتیجه برسد که احتمال افشای راز صفر است.
با توجه به اثرات مشخص کشف شدن راز (اگر کشف می‌شد، مردم نسبت به پزشکان اعتمادشان را از دست می‌دادند)، جراح باید به این نتیجه برسد که انجام چنین عملی اشتباه است.

در حالی که محاسبه ارزش مورد انتظار اغلب ساده است (انجام آن عمل، چندین انسان را نجات می‌داد)، به دلیل سوگیری‌های شناختی انسان (مثلا اینکه شما بیمار من رو به خاطر اهدای عضو، به عمد به قتل رساندید!)، اغلب به درستی استفاده نمی‌شود یا حتی اصلاً اعمال نمی‌شود.
«غفلت احتمالی» یک سوگیری شناختی است که افراد نسبت به «عدم قطعیت‌ها» نشان می‌دهند، به‌ویژه «احتمالات کوچک»، که تمایل دارند یا به طور کامل از آن‌ها غفلت کنند، یا تا حد زیادی (اغراق) آن را بزرگ کنند.
یک مطالعه با دریافت اینکه مردم برای کاهش خطرات "رویدادهای نادر و پر تاثیر" یا ارزش خیلی زیاد یا بسیار پایین قائل هستند؛ (غفلت احتمالی) را تائید کرد.
ما نیازی به جستجوی شواهدی مبنی بر غفلت جمعی از «رویدادهای نادر و پرتأثیر» نداریم!
اگر قانون اجباری بستن کمربند در خودرو برداشته‌شود، به نظر شما چند نفر حاضرند تا کمربندشان ببیندند؟ (این خود نشان دهنده‌ی این است که مردم از سوگیری شناختی غفلت احتمالی استفاده می‌کنند!) این قضیه اصلا هم جالب نیست!
زیرا «رویدادهایی با احتمال کم و تأثیر زیاد» اغلب دارای ارزش‌های مورد انتظار بزرگ، اعم از منفی یا مثبت هستند، این دام در تفکر انسان نگران‌کننده است!
این نشان می‌دهد که انسان اغلب به «ارزش‌های مورد انتظار» حتی فکر هم نمی‌کند!
چه برسد که بخواهد آن را هنگام تصمیم‌گیری به کار ببرد!

سوگیری دیگری که ممکن است بر توانایی افراد در برآورد مقادیر مورد انتظار تأثیر بگذارد، «غفلت از محدوده» است.\     مطالعات نشان داده است که افراد ارزش‌گذاری خود را در تناسب با مقیاس یک مسئله تنظیم نمی‌کنند.
به عنوان مثال، یک مطالعه از سه گروه از افراد در مورد تمایل آن ها به پرداخت هزینه برای نجات 2000 یا 20000 یا 200000 پرنده از غرق شدن در استخرهای نفتی بدون سرپوش پرسیده شد.
میانگین‌های مربوطه 80، 78 و 88 دلار و میانگین پاسخ‌ها همگی 25 دلار بود.
اگر ارزش‌گذاری افراد از برخی نتایج به‌درستی مقیاس‌پذیر نباشد، ارزش‌های مورد انتظار نیز نخواهد بود (یعنی اینجا باید هرکس با توجه به دارایی خود مبلغی را اعلام می‌کرد، ولی همه‌ی آن‌ها پاسخی نزدیک به 25 دلار داده‌بودند).


\phantomsection
\subsubsection*{در انتخاب پروژه‌های خیریه، پروژه‌های (موارد) موثر را انتخاب کنید}
\label{subsubsec:در انتخاب پروژه‌های خیریه، پروژه‌های (موارد) موثر را انتخاب کنید}
\addcontentsline{toc}{subsubsection}{در انتخاب پروژه‌های خیریه، پروژه‌های (موارد) موثر را انتخاب کنید}{\protect\numberline{}}
از آنجایی که مردم معمولاً به جای تحقیق در مورد اثربخشی خیریه، بر اساس انگیزه و احساسات به خیریه می‌پردازند، اغلب از خیریه‌ها و اهداف بی‌اثر حمایت می‌کنند.\     ولی در عوض چیزاهایی نذیر: نوع دوستی مؤثر، یک جنبش جهانی اخیر، بر اهمیت رفتار نوع دوستانه مؤثر، چه در قالب کمک‌های مالی و چه در قالب زمان مهم هستند!

چه خوب است که همین اصل (سراغ کارهایی برویم که اثربخشی بالا دارند) را در هوش مصنوعی پیاده‌سازی کنیم و به اهداف مهم‌تر، اولویت بالاتری بدهیم.


\phantomsection
\section*{اخلاق دئونتولوژیک}
\label{sec:اخلاق دئونتولوژیک}
\addcontentsline{toc}{section}{اخلاق دئونتولوژیک}{\protect\numberline{}}
\textbf{نوشته‌ی کالین مارشال}
\paragraph{}
رویکردهای «دئونتولوژیک» به اخلاق، بر مجموعه‌ای از ایده‌های مرتبط متمرکز است: احترام، استقلال، حقوق، و امتناع از رفتار با انسان‌ها (و شاید سایر موجودات) به‌گونه‌ای که گویی آن‌ها صرفاً چیزها یا ابزارهایی برای رسیدن به اهداف دیگر هستند.
یک تصویر کلاسیک از رویکرد دئونتولوژیک شامل سناریوی زیر است: یک پزشک را تصور کنید که پنج بیمار دارد و هر یک از بیماران، نیاز فوری به اهدای عضو دارند.
یک فرد قابل اعتماد و سالم وارد مطب دکتر می‌شود؛ دکتر می‌تواند فرد سالم را بکشد و اعضای بدن او را برای نجات پنج بیمار برداشت کند.
حتی اگر پزشک بتواند این کار را بدون تشخیص انجام دهد، بسیاری از مردم قضاوت می‌کنند که نباید این کار را انجام دهند.
این قضاوت به راحتی در اصطلاحات دئونتولوژیک، به این صورت بیان می‌شود: عدم احترام از طرف پزشک، به عنوان نقض حقوق فرد سالم، یا به عنوان دکتری که از فرد سالم به عنوان یک چیز صِرف (یک ظرف اندام) استفاده می‌کند.

رویکرد دئونتولوژیک اغلب با رویکردهای «نتیجه‌گرایانه» در تضاد است، که هر عملی را که بهترین نتیجه را به همراه داشته‌باشد توصیه می‌کند؛ یا مثلا اگر در مثلا قبل جزئیات به درستی تکمیل شوند رویکر «فایده‌گرا» پیشنهاد می‌دهد که فرد سالم را برای آن پنج بیماری قربانی کنیم.
زیرا در این رویکرد نتیجه‌ای که حاصل می‌شود، این است که پنج انسان به زندگی برگشتند و فقط یک انسان کشته‌شد.
ولی رویکرد «دئونتولوژیک»، از حق انسان سالم دفاع می‌کند.
با این حال، در عمل، احکام رویکردهای اخلاقی «دئونتولوژیک» و «نتیجه‌گرایانه» غالباً منطبق هستند.
به هر حال، در هر نسخه واقع بینانه‌ای از پرونده دکتر، هیچ تضمینی وجود ندارد که قتل مخفی بماند.

توجه به این موضوع، نتیجه‌گرایی، فاکتورگیری (کنارگذاری) ریسک‌های بزرگی را توصیه می‌کند، مانند کاهش اعتماد به متخصصان پزشکی (که در نتیجه افراد بیمار به دنبال کمک لازم نمی‌گردند) و تأثیر روان‌شناختی مخرب احتمالی بر پزشک (که گناه و آسیب‌های روحی ممکن است آینده آن‌ها را مختل کند).
در نتیجه چنین ملاحظاتی، بسیاری از نتیجه‌گرایان معتقدند که اگر مردم عموماً از منظر دئونتولوژیک به تصمیم‌گیری بپردازند، بهترین پیامدها تضمین می‌شود.
به همین دلیل، می‌توانیم انتظار داشته‌باشیم که بسیاری از ارزیابی‌های «دئونتولوژیک» با ارزیابی‌های «نتیجه‌گرا» (و سایر موارد) همخوانی داشته‌باشند، حتی اگر رویکردهای مختلف بر عوامل متفاوتی تأکید کنند.

مفهوم اصلی دئونتولوژیک احترام، همراه با دو مفهومی است که از احترام بیرون می‌آیند: بی‌طرفی و امتناع از دیگران به عنوان ابزار صِرف یا چیز (منظور نگاه ابزاری به آدم‌ها است).
در اینجا می توانیم به اختصار هر یک از این موارد را بررسی کنیم.
انواع مختلفی از احترام وجود دارد، اما شکل مربوط به احترام اخلاقی توجه جدی‌ای، به نیازها و پروژه‌های دیگران است.
چنین احترام اخلاقی‌ای می‌تواند و البته باید اغلب بر عمل تأثیر بگذارد: اگر ما به طور جدی نیازهای کسی را در نظر بگیریم، معمولاً به گونه‌ای عمل نمی‌کنیم که آن نیازها را تضعیف کنیم.
با این حال، حتی زمانی که اقدامی نیز صورت نگیرد، ممکن است شکست‌هایی در احترام وجود داشته باشد، مانند خندیدن بی‌احترامانه به شکست‌های دیگران آن هم در صورتی که به آن آگاه نباشیم.
رفتار اولیه‌ی ما با دیگران، به ندرت با احترام همراه است (اولین رفتار ما همیشه محترمانه نیست).
در عوض، ما بی‌احترامی را ترجیح می‌دهیم و سعی می‌کنیم که بر اهداف و نیازهای خودمان ترمرکز کنیم تا اینکه بخواهیم نیازهای دیگران را در اولویت قرار دهیم؛ به این رفتار «جانبداری» می‌گویند.
یعنی اهداف خودمان را بر دیگران ترجیح دهیم و برایشان ارزش بیشتری قائل شویم.
مثلا اگر یک پلتفرم شبکه‌ی اجتماعی، تنها با هدف به حداکثر رساندن سود، کاربران خود را به شکل‌های تعامل مضر ترغیب کند، آن‌ها با کاربران خود به عنوان وسیله برای دستیابی به سود رفتار می‌کنند (برای اطلاعات بیشتر به تفسیر مورد ۶ - \hyperref[subsec:مورد۶ - بدافزار ذهنی: الگوریتم‌ها و معماری انتخاب]{\mbox{«بدافزار ذهنی»}} مراجعه کنید).
به طور مشابه، اگر یک مزرعه یا کارخانه با حیوانات به‌عنوان منابع صِرف گوشت رفتار کند، آن‌ها را صرفاً وسیله می‌داند (به تفسیر مورد ۷ - \hyperref[subsec:مورد۷ - هوش مصنوعی و موجودات غیر انسان]{\mbox{«حیوانات و هوش مصنوعی»}} مراجعه کنید).
چنین نگرشی به منزله‌ی شکست کامل احترام است.
صرف‌نظر از اینکه دیدگاه دئونتولوژیک خوب است یا نه، مردم به طور پیش‌فرض به مسائلی مانند: احترام، حقوق و بی‌طرفی اهمیت می‌دهند.


\phantomsection
\section*{اخلاق فضیلت}
\label{sec:اخلاق فضیلت}
\addcontentsline{toc}{section}{اخلاق فضیلت}{\protect\numberline{}}
\textbf{نوشته‌ی جان هکر رایت}
\paragraph{}
اخلاق فضیلت رویکردی به اخلاق یا به عبارت دقیق‌تر، خانواده‌ای از رویکردها است یک انسان برای خوب زیستن به آن نیاز دارد.
این به ما می‌گوید که حالات خوب شخصیت به نام فضیلت را ایجاد و نشان دهیم، و از ایجاد و نشان دادن حالات بد شخصیت به نام رذایل اجتناب کنیم.
برجسته‌ترین رشته‌ی اخلاق فضیلت در آکادمی غرب امروز توسط فیلسوف یونان باستان ارسطو (384-322 قبل از میلاد) ارائه شده است، اما نسخه‌های زیادی از اخلاق فضیلت وجود داشته و دارد.
از این رو، برای مثال، می توان نسخه‌های کنفوسیوس و بودایی از اخلاق فضیلت را یافت.
دیدگاهی که در مورد آنچه در ادامه می‌آید توضیح خواهم‌داد اخلاق فضیلت ارسطویی است.
وقتی به یک فرد خوب فکر می‌کنید، ممکن است به فردی با ویژگی‌هایی مانند شجاعت، شفقت، صداقت و مانند آن فکر کنید.
این‌ها فضایل فرضی است.
هر ویژگی‌ای که فکر می‌کنیم کسی برای خوب زندگی کردن در حوزه خاصی از زندگی انسانی نیاز دارد، تصور ما از فضایل را شامل می‌شود.
در حالی که فهرست قطعی از فضایل وجود ندارد، همگرایی قابل توجهی بر سر ویژگی‌هایی مانند شجاعت، صداقت، عدالت و خرد وجود دارد.
اخلاق‌گرایان فضیلت می‌کوشند تا معیار درستی و نادرستی در عمل را از فضایل یا فرد نیکوکار استخراج کنند.
یکی از فرمول‌بندی‌های برجسته می‌گوید: یک عمل درست است اگر و تنها اگر کاری باشد که یک فرد با فضیلت یا شخصیت، انجام می‌دهد.
توجه داشته‌باشید که حتی اگر خودمان فاضل نباشیم، می‌توانیم از این امر پیروی کنیم، به شرط آنکه سطحی از بینش نسبت به کاری که فاعل با فضیلت انجام می‌دهد و خویشتن‌داری کافی برای انجام آن‌گونه که فرد با فضیلت عمل می‌کند، داشته‌باشیم.
اگر خواسته‌های ما بیش از حد بی‌نظم باشد [ضد و تقیض باشد، مثلا طرفداری از فمینیست به دلیل اینکه ما فرد روشن‌فکری هستیم یا به روشن‌فکران احترام می‌گذاریم]، ممکن است نتوانیم با نیات نیکو عمل کنیم و حتی ممکن است در نتیجه تلاش برای عمل به عنوان یک عامل نیکوکار، بدتر عمل کنیم!
در این صورت، اختیار اخلاقی ما به دلیل ضعف اراده به خطر بیافتد.
هدف ما همچنان این است که بتوانیم همانطور که عامل فاضل عمل می‌کند، عمل کنیم.

ممکن است قوانینی وجود داشته‌باشد که کلیات، الگوهای عمل، و ویژگی‌های استدلالی افراد با فضیلت را به تصویر بکشد، اما نمی‌توان آن‌ها را بدون تفکر به کار برد.
به عبارت دیگر، سطحی از درک اخلاقی برای اعمال آن‌ها ضروری است.
این ممکن است نقطه ضعف نظریه به نظر برسد، اما از سوی دیگر، نظریه‌های رقیب خود را متعهد به دیدگاه‌های عمیقاً ضد شهودی و گاه از نظر اخلاقی آزاردهنده درباره کنش درست بر اساس قوانین استثنایی می‌دانند: برای مثال، دیدگاه دین‌شناختی «امانوئل کانت» به طرز بدنامی به موضعی استثنایی متعهد است.
هرگز دروغ نگفتن، حتی اگر این کار باعث نجات جان انسان‌ها شود.
در مقابل، اخلاق‌دانان فضیلت ممکن است معتقد باشند که نیاز انسان به روابط اعتماد، صداقت را به یک فضیلت تبدیل می‌کند و در عین حال ادعا می‌کنند که ما می‌توانیم تعهد خود را به صداقت حفظ کنیم و در عین حال شرایطی را که دروغ را می‌طلبد مجاز بدانیم.
به عنوان مثال، اگر از ما اطلاعات شخص خاصی را به منظور قتل خواستند، دروغ گفتن مناسب است.
فقدان قوانین استثنایی نیز ممکن است یک مزیت برای اخلاق فضیلتی در برخورد با فناوری‌های نوظهور باشد.

از آنجایی که فضایل در مرکز اخلاق فضیلت قرار دارند، بسیار مهم است که بدانیم آن‌ها چیستند.
برخی از فضایل برتری امیال و احساسات ما هستند، در حالی که برخی دیگر مانند حکمت عملی، در درجه اول برتری‌های فکری هستند.
به عنوان مثال، شجاعت، به خواست ما به امنیت مربوط می‌شود و زمانی نشان داده می‌شود که احساس ترس و اعتماد به نفس ما به گونه‌ای باشد که فقط در مواجهه با چیزی که واقعاً خطرناک است، احساس ترس کنیم.
ارسطو ایده‌ی فضیلت را با توسل به «آموزه پست» معروف خود توضیح داد.
در یک انسان شجاع، احساس ترس و اطمینان در حالتی میانی بین افراط و کمبود قرار دارد.
کسی که احساس ترس بیش از حد می‌کند، از خطر فرار می‌کند و نمی‌تواند به چیزی ارزشمند دست یابد.
ما به این افراد برچسب ترسو می‌زنیم زیرا آن‌ها رذیلت بزدلی را نشان می‌دهند.

کسی که احساس ترس بسیار کمی دارد ممکن است بی پروا عمل کند و در تلاش‌های بیهوده‌ای که باید از آن اجتناب می‌شد با جراحت یا مرگ مواجه شود.
ویژگی رویکرد ارسطویی این است که ترس، در کنار سایر احساسات، چیزی است که برای خوب زیستن ضروری است.
از این گذشته، وقتی احساس ترس می‌کنم، ارزش زندگی و تمامیت جسمی‌ام را به گونه‌ای ثبت می‌کنم که انگیزه‌ای برای عمل ایجاد کند.
با این حال، من ممکن است برای زندگی و تمامیت جسمی خود بیش از حد ارزش قائل شوم.
از نظر ارسطو، چیزهای مهمتری از زندگی و تمامیت جسمانی من وجود دارد، مانند آزادی شهرم و امنیت دوستان و خانواده‌ام.
از این رو، از نظر او، در صورت وجود شانس غیرمعمول برای دستیابی به چنین هدفی، خطر مرگ چیز خوبی است.
جنبه دیگری از دیدگاه ارسطو این است که شخص نمی‌تواند شجاعت نشان‌دهد مگر اینکه برای رسیدن به هدفی ارزشمند با ترس روبرو شود.
دزدی که به خاطر دزدی با خطر روبرو می‌شود، شجاع نیست.
اگرچه شخصیت آن‌ها به گونه‌ای است که مستعد احساس ترس نیستند، اما این حالت شخصیتی در آن‌ها برتری ندارد.

شرارت آن‌ها (دزدان) در حوزه دیگری، توانایی آن‌ها را برای رفتار شجاعانه تضعیف می‌کند.
این جنبه دیگری از رفتار شناسان ارسطو است: او از ایده‌ای به نام «وحدت فضایل» دفاع می‌کند که در قوی‌ترین شکل خود بیان می‌کند که برای داشتن یک فضیلت باید همه آن‌ها را داشته‌باشیم.
به بیان دیگر، ایده این است که هر رذیله‌ای، توانایی نشان دادن هر فضیلتی را تضعیف می‌کند.
با فرض اینکه دولت‌هایی واسطه بین فضیلت و رذیلت وجود دارد، این امر فضایی را برای کمتر از فضیلت کامل بودن در برخی زمینه‌ها باز می‌کند بدون اینکه لزوماً فضیلت ما را در سایر زمینه‌ها تضعیف کند.
با ماندن در فضیلت شجاعت به عنوان مثال، می توانیم تعجب کنیم که آیا شجاع بودن خوب است؟ به هر حال، اگر مستلزم این باشد که به خاطر دولت شهرم جانم را به خطر بیندازم، شاید بهتر باشد که ترسو باشم.
اما توجه داشته‌باشید که این دیدگاه بزدلانه جهان را می‌پذیرد: اینکه به هر قیمتی زنده ماندن بهتر است.
انسان شجاع دنیا را متفاوت می بیند: بقا وقتی به قیمت آزادی شهر خود یا مرگ یا بردگی دوستان و خانواده‌اش تمام شود، خوب نیست.

پس آیا، ما در، کنارِ هم قرار گرفتن این دو دیدگاه گیج شده‌ایم یا اینکه دیدگاه شخص شجاع تطابق دارد؟ من معتقدم که دیدگاه افراد شجاع برتر است زیرا شجاعت یک ویژگی است که انسان برای زندگی خوب در دنیای خطر به آن نیاز دارد.
ما انسان ها باید بتوانیم اهداف را، حتی در مواجهه با خطرات به پیش ببریم.
این دیدگاه نسخه‌ای از اخلاقی است که بسیاری از ارسطویی‌ها آن را پذیرفته‌اند: اینکه خوبی در انسان، تابعی از نوع حیوانی است که آن‌ها هستند (که این حرف را فقط ارسطویی‌ها می‌گویند).
فضائل قوای عقلانی و اشتهایی انسان را کامل می‌کند و این امری عینی است که صفات آن چنین است.

ارسطو در عصری با ساختار اجتماعی بسیار متفاوت و همچنین با فناوری‌های متفاوت زندگی می‌کرد.
یقیناً امروزه هیچ یک از اخلاق‌شناسان فضیلت ارسطویی، نظرات او را بدون تعدیل نمی‌پذیرد.
تأکید بیش از حد ارسطو بر فضیلت رزمی شجاعت در دیدگاه‌های سیاسی او، باعث چسباندن انگ زن‌ستیزی و نژادپرستی در زمان خود شد.
اما چارچوب فلسفی او همچنان بینش را به همراه دارد.
اخلاق فضیلت ارسطویی در پرداختن به سؤالات فناوری و علم داده، بر بررسی تأثیر فضیلت بر شخصیت ما تأکید می‌کند: چگونه استفاده از یک فناوری جدید بر تمایلات و تفکر ما تأثیر می‌گذارد؟ اگر یک فناوری ما را وادار می‌کند چیزی به عنوان ویژگی یک عامل شرور فکر یا احساس کنیم، پس این زمینه ای برای انتقاد اخلاقی از فناوری است.
از این رو، تمرکز بر این است که چگونه با فناوری زندگی می‌کنیم.
ما مجبور نیستیم برای ایجاد شک و تردیدهای اخلاقی در مورد یک فناوری، تأثیر چشمگیری بر جامعه یا نقض وظایف داشته‌باشیم.
ما می‌توانیم با بررسی تحریف‌ها و تأثیرات آن بر افکار و احساسات خود به نقد اخلاقی فناوری نزدیک شویم (منظور اینکه فناوری چه تأثیرات بدی بر روی اخلاقیات ما داشته‌است).
فناوری‌های جدید ممکن است خواسته‌های اخلاقی جدیدی از ما ایجاد کنند.
در چنین مواردی، این پرسش مطرح می‌شود که آیا فضیلت جدیدی لازم است یا صرفاً تفکر در مورد یک فضیلت سنتی در بستری جدید است.
نظر من این است که تمایل بر این است که جنبه‌های فضایل سنتی را دوباره پیکربندی کنند، و انجام این کار ضرری ندارد و ممکن است فایده‌ای داشته باشد، زیرا ممکن است به ما کمک کند تا با دقت بیشتری در مورد موقعیت‌هایی که با آن روبرو هستیم فکر کنیم.
به طور خلاصه، اخلاق فضیلت ارسطویی چارچوبی انعطاف‌پذیر برای اندیشیدن در مورد اینکه چقدر با فناوری‌های جدید زندگی می‌کنیم فراهم می‌کند، و نیازی نیست که آن را محکم با دیدگاه‌های باستانی ارسطو در مورد فضایل گره بزنیم.

اگر فرض شود که ما به‌عنوان افراد به تنهایی می‌توانیم ویژگی‌هایی را که برای خوب زندگی کردن در هر شرایطی به آن‌ها نیاز داریم، توسعه دهیم و از خود نشان دهیم، اخلاق فضیلت نادرست درک می‌شود.\ درعوض، اخلاق فضیلت، مربوط به سنجش شرایط اجتماعی است که برای خوب زیستن انسان‌ها ضروری است.
این امر به ویژه در درنظر گرفتن تأثیر فناوری‌های جدید بسیار مهم است.
آن‌ها ممکن است توانایی ما را برای تطبیق خواسته‌هایمان با اهداف آگاهانه‌مان تضعیف کنند (یا به‌عنوان خوش‌بین‌تر، تقویت کنند)، و در نتیجه تلاش‌های ما برای توسعه فضایل را تضعیف کنند.
از دیدگاه ارسطویی، رشد فضایل مستلزم فرآیند عادت کردن است، یعنی فرآیندی از عمل به گونه‌ای که فاعل نیکوکار عمل می‌کند، شاید بر خلاف تمایلات ما، تا زمانی که از عمل به آن طریق لذت ببریم و بتوانیم آن را با اطمینان انجام دهیم (پس ارسطو میگوید که باید به رفتارهای خوب و نیکو، عادت کنیم).


\newpage


\phantomsection
\section*{اخلاق آفریقایی}
\label{sec:اخلاق آفریقایی}
\addcontentsline{toc}{section}{اخلاق آفریقایی}{\protect\numberline{}}
\textbf{نوشته‌ی جان مورانگی}
\paragraph{}
در ادامه، باید انتظار دیدِ موقتی درباره‌ی اخلاق آفریقایی داشت.
موقتی بودن اهمیت دارد زیرا جایی برای دیدگاه‌های دیگر باقی می‌گذارد.
علاوه بر این، خواننده را متوجه این واقعیت می‌کند که آنچه در مورد اخلاق آفریقایی گفته می‌شود، همه‌ی آن نیست.
چیزهای بیشتری برای گفتن وجود دارد؛ که از آن صرف نظر می‌کنم.
اگر بخواهیم در مورد درک اخلاق آفریقایی عدالت را رعایت کنیم، کنار گذاشتن نژادپرستی بینش مهمی است.
اخلاق آفریقایی مانند هر شاخه‌ی دیگری از اخلاق یک اخلاق منحصر به فرد است.
نباید آن را با هیچ شاخه دیگری از اخلاق اشتباه گرفت.
اخلاق، چه افریقایی یا غیرافریقایی، چه خاص و چه جهانی، در مورد رفاه است.
در جوامع بومی آفریقا، رفاه اجتماعی، رفاه اجتماعی است.
این بهزیستی است که جایگاهی برای رفاه فردی و همچنین رفاه گروهی دارد (منظور از رفاه که امروزه استفاده می‌شود، پول و جایگاه مادی است).
یک جمله‌ی معروف در اخلاق آفریقایی و اوبونتو وجود دارد که برایتان آورده‌ام: ما هستیم، پس من هستم، این نشان‌دهنده‌ی این است که اخلاق آفریقایی برای ما (جمع انسان‌ها) ارزش بالایی قائل است.
در اخلاق اوبونتو که زیرشاخه‌ی اخلاق آفریقایی است، هیچ‌وقت ارزش یک فرد، بالاتر از ارزش یک جمع نیست.

این مهم است که به خود یادآوری کنیم که اخلاق آفریقایی تابع قوم نگاری یا قوم شناسی نیست، این اخلاق قومی و قبیله ای نیست.
همچنین این قضیه را باید به صورت محکم بیان نمود که کاشفان اروپایی در تاریخ مدرن میگفتند که آفریقایی‌ها وحشی هستند!
این باور کاملا غلط و برخواسته از نژادپرستی است!

از آنجا که اخلاق در سعادت جامعه دخیل است، به نظر می‌رسد که جامعه شناسی در مطالعه اخلاق ضروری است.
همانطور که جامعه شناسی مطالعه جامعه است، مطالعه اخلاق نیز در جامعه شناسی گنجانده شده‌است.
علاوه بر این، از آنجایی که جامعه از نظر سیاسی امنیت دارد و منافع آن توسط دولت (سیاسی) تبلیغ و پیگیری می‌شود، اخلاق اساساً سیاسی است.
به گونه ای دیگر، اخلاق تابع جامعه شناسی سیاسی است.
در اخلاق متعارف اروپایی-غربی، معماری چندلایه اخلاق به ندرت به رسمیت شناخته می‌شود.
در بافت بومی آفریقا، این معماری به رسمیت شناخته شده‌است.


\phantomsection
\section*{اخلاق بودایی}
\label{sec:اخلاق بودایی}
\addcontentsline{toc}{section}{اخلاق بودایی}{\protect\numberline{}}
\textbf{نوشته‌ی پیتر هرشوک}
\paragraph{}
اخلاق می‌تواند شامل همه چیز باشد، از تبیین چیزی که به طور ایده‌آل در یک فرد «خوب» دخیل است، تا معنای عملی نمایندگی «قابل قبول» در یک حرفه یا شهروندان یک ملت یا جهان.

من به اخلاق به صورت عملیاتی برخورد می‌کنم و آن را حداقل به‌عنوان هنر ارزشیابی اصلاح مسیر انسانی تعریف می‌کنم: هنر اعمال هوشمندانه نتایج حاصل از تبعیض مشترک و کیفی بین ارزش‌ها، اهداف و علایق و ابزارهای ما برای تحقق آن‌ها.
برای من، این هنری است که به طور اساسی با شرح و بسط معاصر مفاهیم و اعمال بودایی آشنا شده است.

بودیسم حدود 2600 سال پیش در دامنه‌های هیمالیا در جنوب آسیا ظهور کرد، تقریباً همزمان با سنت‌های فلسفی و سیاسی جهان مدیترانه و سینیتی.
آن سنت‌ها با پرسش‌های بنیادینی دست و پنجه نرم می‌کردند: چه چیزی واقعی است؟ چی خوبه؟ جایگاه انسانیت در کیهان چیست؟ و جامعه چگونه باید اداره شود؟ بودیسم در پاسخ درمانی (به جای نظری) به دو سؤال متفاوت، اما به همان اندازه اساسی، پدید آمد.
علل و شرایط ابتلا به دعا یا رنج و درگیری و گرفتاری چیست؟ و با چه وسیله‌ای می توانیم این علل و شرایط را از بین ببریم؟ پاسخ بوداییان به این سؤالات بر دو بینش کلیدی استوار است.
اولاً، همه چیز به طور متقابل به وجود می‌آید و ادامه می یابد.
به طور قوی بیان می‌شود که رابطه گرایی اساسی تر از چیزهای مرتبط است.
همه چیز تابعی از تمایز رابطه‌ای است، و هر چیز در نهایت همان چیزی است که برای دیگران معنا می‌کند.
ثانیاً، کیهان ما خودسازمانده و دارای ساختار کرمی است.
این کیهانی است که در آن الگوهای ثابت ارزش‌ها، نیات و اعمال منجر به الگوهای همخوانی از نتایج و فرصت‌های تجربی می‌شود.

هدف هنر بودایی اصلاح سیر انسانی، تحقق آزادی از درهم تنیدگی‌های رابطه‌ای است که دخخا ایجاد می‌کند، عمدتاً از طریق حل تعارضات بین ارزش‌ها، نیات و اعمال ما.
این بستگی به ارزیابی انتقادی عادات فکر، گفتار، و رفتار، و تحقق آزادی توجه و آزادی نیت مورد نیاز برای تجدید نظر، مقاومت، یا انحلال آن عادات در صورت لزوم دارد تا دیگر توسط درهم تنیدگی‌های کارمایی و حضور اجباری محدود نشوند.
به طور قابل توجهی، هدف تمرین بودایی (هدف نیروانا) تجویز یا تعریف نشده است.
در عوض، به طور سنتی به صورت استعاری به عنوان خنک‌کننده یا خاموش‌کننده آتش ولع، بیزاری، و جهل تلقی می‌شد.
این پیامدهای مهمی برای اخلاق بودایی دارد.
به طور خلاصه، اخلاق بودایی هدف یا مقصد نیست.
یک هنر بی پایان و بداهه است.
اخلاق بودایی را می‌توان با برخی توجیهات، شامل عناصری از رویکردهای مبتنی بر فضیلت، وظایف (دئونتولوژیک) و مبتنی بر پیامد (فایده‌گرا) به اخلاقی دانست که در فلسفه غرب غالب شده‌اند، و همچنین رویکردهای مراقبت محوری مانند فمینیستی.
با این حال، هستی شناسی رابطه ای بودایی به طور مشخص توجه ارزیابی را از عوامل اخلاقی، بیماران و اعمال مستقل و به سمت کیفیت رابطه ای سوق می‌دهد.
علاوه بر این، در حالی که تأکید بودیسم بر فضیلت‌گرایی رابطه‌ای، اخلاق بودایی را متعهد به شرایط خاص می‌کند، با اخلاق موقعیتی غربی که اعمال را بر اساس نتایج نزدیک یا کوتاه‌مدت ارزیابی می‌کند، متفاوت است.
آنچه از نظر اخلاقی اهمیت دارد صرفاً پیامدهای فوری یک عمل نیست، بلکه پیامدهای رابطه‌ای میان‌مدت و بلندمدت اجرای عمدی مجموعه‌های ارزش‌های خاص و شکل‌دهی آن‌ها به فرصت‌های ارادی و نیز نتایج تجربی است.


\phantomsection
\section*{اخلاق بومی و فطری: کنش‌ها به مثابه تعامل}
\label{sec:اخلاق بومی و فطری: کنش‌ها به مثابه تعامل}
\addcontentsline{toc}{section}{اخلاق بومی و فطری: کنش‌ها به مثابه تعامل}{\protect\numberline{}}
\textbf{نوشته‌ی جوزف لن میلر و آندریا سالیوان کلارک}
\paragraph{}
پاسخ به این سؤال که یک نظریه اخلاقی بومی چگونه است دشوار است.
اول، مشکل «پان ایندیانیسم» وجود دارد.
با توجه به تفاوت‌هایی که بین قبایل وجود دارد، اندیشیدن به مردم «بومی» به‌عنوان یک گروه همگن مشکل‌ساز است.
دوم، از نظر تاریخی، اندیشه فلسفی مردم بومی به طور جدی دست کم گرفته شده‌است.
اکثر متفکران غربی فرض کرده‌اند که مردم بومی آنقدر بدوی یا حتی «وحشی» بودند که نمی‌توانستند در مورد موضوعات یا پرسش‌های انتزاعی تأمل کنند.
این تاریخ تأثیرات ماندگاری بر فلسفه بومی دارد.
نه تنها ایده‌های بومی، حتی بنیادی‌ترین آن‌ها، باید بر اساس استانداردهای غربی «توجیه» شوند، بلکه این ایده‌ها باید در زمینه‌ای غیر از آنچه در آن شکل گرفته‌اند، توضیح داده‌شوند.
همانطور که گفته شد، یکی از تمرکز مشترک مهم اخلاق بومی، به هم پیوستگی همه چیز است (مانند مردم، زمین، حیوانات غیر انسانی، نسل‌های گذشته و آینده و غیره).
کیهان موجودی زنده است و درک می شود که در «گذار دائمی» است.
این موضوع زمینه ای را برای مردم بومی فراهم می‌کند که «بر اساس اصول تعادل هماهنگی عمل می‌کند».
مردم در اجتماع و روابط متولد می شوند.
این‌ها شامل روابط غیرانسانی مانند ارواح، صخره‌ها، رودخانه‌ها، اعضای گونه‌های حیوانی غیرانسانی و غیره می‌شوند.
هر موجودی که ما با آن رابطه داریم متفاوت است، و بنابراین اقدامات ما نسبت به روابطمان نیز متفاوت خواهدبود.
به جای ارائه اصول جهانی برای هدایت رفتار، مفاهیم کلیدی وجود دارد که پایه و راهنمایی را برای تصمیم گیری اخلاقی فراهم می‌کند.
این مفاهیم عبارتند از هماهنگی، متقابل، سپاسگزاری و فروتنی.
درک چگونگی ارتباط این مفاهیم با یکدیگر می‌تواند به درک بهتر نحوه اجرای این مفاهیم در زمینه‌های مختلف کمک کند.
روش صحیح زندگی، و عمل، سپس با آنچه ما از روابط خود و ارتباط ما با این مفاهیم می‌دانیم، آگاه می‌شود.

هماهنگی زمانی وجود دارد که بین مبادلات و تعاملات با محیط اطراف فرد تعادل وجود داشته‌باشد.
تعادل و هماهنگی، ویژگی‌های دنیایی است که ما در آن متولد شده‌ایم، راهنمایی برای اطمینان از رفاه روابط ما و خودمان است.
با توجه به وابستگی متقابل و روابط بین همه‌چیز، هر تعاملی بر رفاه یک فرد و محیط اطراف او تأثیر می‌گذارد.
به عبارت دیگر، برای هر کنش، واکنشی است.
برای ایجاد تعادل در این تعاملات، یک فرد باید بداند که چگونه متقابلا عمل کند.
تعامل متقابل می‌تواند اشکال مختلفی داشته‌باشد (یعنی یک روش "درست" منحصر به فرد برای انجام متقابل وجود ندارد)، اما باید متناسب با موجودی باشد که فرد با او در تعامل است.
هدف متقابل ایجاد تعادل در روابط است تا همه موجودات درگیر بتوانند به طور مسالمت آمیز با هم زندگی کنند.
برای زندگی مسالمت آمیز با محیط اطراف، و رفتار متقابل مناسب، باید با عشق، سپاسگزاری و فروتنی رفتار کرد.

با در نظر گرفتن این مفاهیم، برای هر کنش (فعل) خاص باید سؤالات زیر را در نظر گرفت: چه عملی هماهنگی ایجاد می‌کند؟ چگونه باید آنچه را که به من داده‌اند جبران کنم؟ آیا با عشق، سپاسگزاری و فروتنی رفتار می‌کنم؟ توجه داشته باشید، پاسخ به این سؤالات به شدت به محیط و زمینه فرد بستگی دارد.
پاسخگویی مناسب به این سؤالات مستلزم داشتن شناخت دقیق از محیط و روابط خود است.
به عنوان مثال، دانستن چگونگی ایجاد هماهنگی (یعنی دانستن نحوه انجام رفتار متقابل) در رابطه با زمین مستلزم دانستن جزئیات دقیق در مورد خاک، زندگی گیاهی، بدنه‌های آبی، الگوهای آب و هوا، وابستگی متقابل بین گیاهان و حیوانات در منطقه است و غیره.
برخی از مفاهیم سیاسی تا حدی به عنوان وسیله ای برای حفظ شیوه‌های زندگی که در حضور استعمار شهرک نشینان حول این مفاهیم شکل می گیرد، نقش برجسته تری در اخلاق بومی ایفا کردند.
این شامل مفاهیم حاکمیت و احیاء است.
از آنجایی که تمرکز این مجموعه اخلاق است، مفاهیم اساسی اخلاقی را که حاکی از تصمیم گیری اخلاقی در فلسفه بومی است، در اولویت قرار داده‌ایم.
با این حال، با توجه به اهمیت و الهام‌بخش تبلیغات اخیر در مورد حاکمیت داده‌های بومی، ما از به اشتراک گذاشتن منابعی که نشان می‌دهند چگونه این مفاهیم (حاکمیت و احیا) در جمع‌آوری و استفاده از داده‌های مربوط به مردم بومی استفاده می‌شوند، خودداری می‌کنیم.

«کوکوتای» و «تیلور» اخیراً جلدی را ویرایش کرده‌اند که مقالاتی را در حمایت از «حقوق و منافع ذاتی و غیرقابل انکار مردمان بومی در ارتباط با جمع‌آوری، مالکیت، و کاربرد داده‌های مربوط به مردم، شیوه‌های زندگی و سرزمین‌هایشان» جمع‌آوری کرده‌اند.\ «رودریگز-لون بیر» و «مارتینز» استدلالی را در حمایت از «تغییر موقعیت اقتدار بر داده‌های بومی به مردم بومی» ارائه می‌کنند.
«کارول» و همکاران نمونه‌هایی از اصول \textenglish{\textbf{(CARE)}}  برای حاکمیت داده‌های بومی (منافع جمعی، اختیار کنترل، مسئولیت و اخلاق) را بیان، توصیف و ارائه کنید.

به طور کلی، مردم بومی با فروتنی به این سؤال می پردازند که چگونه خوب زندگی کنند، زیرا می‌دانند که ما تنها بخش کوچکی از جهان هستیم.
ما برای زنده ماندن به رفاه و سخاوت خویشاوندان خود (یعنی همه روابطمان) وابسته هستیم.
اختلال در کار، هرج و مرج، بی نظمی و زوال رفاه بستگان ما ناهماهنگی ایجاد می‌کند و نشان دهنده این است که اعمال ما نادرست است و باید راه خود را تغییر دهیم.



% \textsuperscript{\hyperref[subsec:مورد۷ - هوش مصنوعی و موجودات غیر انسان]{ali}}