%! Author = zamoosh
%! Date = 6/7/23


\chapter{اخلاق تحقیق و روش علمی}
\label{ch:اخلاق تحقیق و روش علمی}
\phantomsection

\begin{quote}
    اخلاق تحقیق و روش علمی برایان وانسینک اجازه نمی‌دهد شکست یک گزینه باشد.
    اگر داده‌های جالبی داشته باشد، به آن ادامه می‌دهد تا زمانی که چیزی پیدا کند، سپس منتشر می‌کند، منتشر می‌کند، منتشر می‌کند.
    \begin{flushleft}
        \textenglish{\textbf{Andrew Gelman, Statistician}}
    \end{flushleft}
\end{quote}

{\setstretch{0.5}
\phantomsection
\section*{"یک ترفند ساده": آزمایشگاه غذا و برند کورنل}
\label{sec:"یک ترفند ساده": آزمایشگاه غذا و برند کورنل}
\addcontentsline{toc}{section}{"یک ترفند ساده": آزمایشگاه غذا و برند کورنل}{\protect\numberline{}}
آیا می‌دانستید اگر در رستوران مورد علاقه خود کنار پنجره بنشینید، 80 درصد بیشتر احتمال دارد که سالاد انتخاب کنید؟ یا اینکه اگر نزدیک میله بنشینید (در نور کم و با پخش موسیقی بلند در پس زمینه) کالری بیشتری مصرف می کنید؟ آیا می‌دانستید افرادی که جعبه‌های غلات خود را بیرون پیشخوان نگه می‌دارند به طور متوسط 21 پوند وزن بیشتری نسبت به کسانی دارند که آن‌ها را در کمد پنهان می‌کنند؟ یا اینکه برند کردن سیب با شخصیت‌های کارتونی محبوب، مانند المو، باعث می‌شود که بچه‌ها با ناهار یکی از آن‌ها را به جای شیرینی انتخاب کنند؟ یا اینکه مردها وقتی خانم ها آن‌ها را تماشا می‌کنند بیشتر غذا می خورند (اما وقتی مردها آن‌ها را تماشا می‌کنند خانم ها کمتر می‌خورند؟ یا اینکه ایجاد یک «پوز قدرت» تأثیر مثبتی بر مصاحبه‌های شغلی، مذاکرات و سایر عملکردها دارد) به‌ویژه برای کسانی که موقعیت اجتماعی پایین‌تری دارند و منابع کمتری دارند؟
}

اگر به همه‌ی یا هر یک از این سؤالات «نه» پاسخ داده‌اید، می‌توانید به خودتان تبریک بگویید، زیرا حق با شماست.
ادعاهای مطرح شده توسط محققان در مطالعات فوق (که همگی زمانی به طور برجسته در رسانه‌ها تبلیغ می‌شدند) قابل تکرار نبودند و از آن زمان پس‌گرفته شدند.
کار ایمی کادی روی ژست‌های قدرتی موضوع دومین سخنرانی پربیننده TED تا به حال بود، و حتی قبل از رد شدن، بخشی از حکمت عامیانه فرهنگی دریافتی ما شد.
ادعاهای دیگر نیز راه خود را به عقل عامیانه علاقه مندان به آخرین اخبار رژیم غذایی و سلامتی (از جمله کسانی که مسئول تصمیم‌گیری در مورد برنامه‌های ناهار مدارس دولتی هستند) باز کرد.
آن‌ها نیز پس از یافته‌های مربوط به تخلفات تحقیقاتی، همه‌ی آن‌ها پس‌گرفته شده‌اند.

این مطالعات محصول «برایان وانسینک» از دانشگاه «کرنل» \textenglish{\textbf{(Cornell University)}}  بود، جایی که او روانشناسی غذا خوردن را در آزمایشگاه غذا و برند «کورنل» خود مطالعه کرد.
«وانسینک»، \textenglish{\textbf{(Food \& Brand Lab)}} را در سال 1997 در دانشگاه «ایلینویز» \textenglish{\textbf{(Illinois)}} تأسیس کرد و در سال 2005 آن را به \textenglish{\textbf{«Ivy League»}} منتقل کرد.
آزمایشگاه \textenglish{\textbf{(Food \& Brand)}} بیشتر بودجه خود را از شرکت‌های مواد غذایی دریافت کرد.
آزمایش‌های «وانسینک» نه تنها از بودجه خوبی برخوردار بودند، بلکه از محبوبیت بالایی برخوردار بودند.
کتاب او با نام «غذا خوردن بی فکر: چرا بیشتر از آن چیزی که فکر می کنیم می خوریم» در سال 2006 در فهرست پرفروش ترین‌های نیویورک تایمز قرار‌گرفت.
فلسفه او کاملاً با حکمت رایج در آن زمان متفاوت بود: وانسینک معتقد بود که به جای اینکه به مردم درباره فواید خوب آموزش دهد، با انتخاب های غذایی و خطرات افراد فقیر، او می توانست مردم را وادار کند تا ترفندها و عاداتی را به کار گیرند که آن‌ها را به سمت بهتر غذا خوردن سوق می‌دهد، بدون اینکه زیاد فکر کنند، یا مجبور باشند به هیچ وجه در مورد انتخاب هایشان منطقی باشند.
او در سال 2015 به «کیرا باتلر» از «مادر جونز» گفت: «میلیون‌ها متخصص تغذیه وجود دارند که به شما می‌گویند به جای شکلات اسنیکرز، یک سیب بخورید، اگر واقعاً می‌خواهیم بهتر غذا بخوریم، باید مغزمان را فریب دهیم».

با این حال، دانشمندان دیگر شروع به ابراز نگرانی در مورد روش‌های تحقیق وانسینک کردند، از جمله «تناقض داده‌ها، غیرممکن‌های ریاضی، اشتباهات، تکراری‌ها، اغراق‌ها، تفسیرهای تعجب‌آور، و مواردی از سرقت ادبی خود در 50 مطالعه او» که بسیاری از آن‌ها نشان داده‌شده و از آن زمان پس‌گرفته شده است.
این‌ها شامل چندین مقاله است که توضیح می‌دهد چگونه ارائه جذاب غذاهای سالم در کافه تریاهای مدرسه باعث تشویق دانش آموزان به انتخاب میوه و سبزیجات بیشتر می‌شود.
برنامه‌های مبتنی بر نشریات لغو شده وانسینک در 30000 مدرسه ایالات متحده اتخاذ شده است که میلیون‌ها دلار بودجه دولتی برای جنبش ناهارخوری‌های هوشمندتر جذب کرده‌است.
این برنامه‌ها عمدتاً شامل دادن نام‌های تند و جذاب و برندهای رنگارنگ به غذای سالم بود، مانند «آب‌میوه‌گیر پرتقال»، «تلفن میمون یا موز»، «سیب لذیذ»، «برش‌های خنک خیار» و «پای شیرین سیب‌زمینی‌ها».

شکاف‌های تحقیق در اوایل قابل مشاهده بودند، اما به دلیل پست وبلاگی توسط خود وانسینک (که باید یکی از پیامدترین اقدامات غرورآفرین در تاریخ علم باشد)، به اوج خود رسیدند.
در وبلاگ، وانسینک یک مجموعه داده‌ی اصلی را که طی چند هفته مشاهده در یک رستوران پیتزا در شمال نیویورک جمع‌آوری شده‌است، مورد بحث قرار می‌دهد.
او خاطرنشان می‌کند که طرح تحقیق اولیه به نتیجه نرسید، بنابراین او به دنبال استخراج داده‌ها برای برخی از نتایج تحقیقات جدید "خوب" بود.
او سپس به شدت از پسادکتری (با پول) خود به دلیل امتناع از کار با داده‌ها انتقاد کرد، در حالی که یک دکتر (بدون حقوق) از ترکیه، داده ها را استخراج کرد و در نهایت پنج مقاله مختلف منتشر کرد (که البته اکنون "مقاله‌های پیتزا" (اسم مقاله)، بدنام هستند.
وانسینک به خلاقیت و ابتکار محقق ترک در تهیه‌ی این همه داده تبریک گفت و اظهار داشت: «با اینکه من با پسادکتری دانشگاه را ترک کردم، ولی به اندازه‌ی یک‌چهارم شما مقاله چاپ کردم».
وانسینک به محقق ترکی، غبطه می‌خورد.
تیم «ون درزی» از دانشگاه «لیدن در هلند»، یکی از اولین دانشمندانی بود که پست وبلاگ وانسینک را خواند و از سوء رفتار احتمالی در «مقاله‌های پیتزا» سخن گفت.
مطالعات روی مقاله‌های پیتزا پس‌گرفته‌شده در یک رستوران بوفه‌ای به نام رستوران ایتالیایی \textenglish{\textbf{Aiello}} در حدود 30 مایلی کرنل انجام شد.
نمونه شامل حدود 130 بزرگسال بود که در یک دوره دو هفته‌ای در رستوران غذا خورده بودند.

نویسندگان خاطرنشان کردند که عدم بیان اینکه داده‌ها، همگی از یک مطالعه میدانی که قبلاً منتشر شده‌است، جمع‌آوری شده‌اند، باعث می‌شود که اعتبار آزمایشات از بین برود و درضمن این نکته، در هیچ‌یک از مقاله‌ها چاپ نشده بود!
هنگامی که آنها از وانسینک درخواست کردند، از دسترسی به داده‌های اصلی نیز محروم شدند.
آن‌ها خاطرنشان کردند که حجم نمونه بین مقالات ناسازگار است، و نشان می‌دهد که برخی از شرکت کنندگان در برخی از مقالات گنجانده شده‌اند، و در برخی دیگر حذف شده‌اند!
«ون درزی» همچنین به چندین اشتباه دیگر در این مقاله اشاره کرد:

\begin{quote}
    انواع خطاها عبارتند از: اندازه‌های نمونه غیرممکن در داخل و بین مقالات، آمارهای آزمایشی محاسبه‌شده و/یا گزارش‌شده نادرست و درجات آزادی، و تعداد زیادی از میانگین‌های غیرممکن و انحرافات استاندارد.
    در مجموع، ما تقریباً 150 تناقض و عدم امکان را در این چهار مقاله شناسایی کردیم.
    در مجموع، این مشکلات اعتماد به نتیجه گیری نویسندگان را دشوار می‌کند.
\end{quote}

در ابتدا، «وانسینک» اشتباهات را جزئی و انتقادات را به عنوان "زورگویی سایبری" رد کرد، اما درخواست‌ها برای تحقیق کامل در مورد تحقیقات او افزایش یافت.
«اندرو گلمن»، آماردان برجسته در دانشگاه کلمبیا، سپس در یک پست وبلاگی تند، وانسینک را صدا زد.
گلمن اظهار داشت: «آنچه برایان را توصیف می‌کنید شبیه به \textenglish{\textbf{(p-hacking)}} و \textenglish{\textbf{(HARKing)}} است.
مشکل این است که اگر فرضیه اصلی شما احتمالی کمتر از ۵۰ درصد داشت، احتمالاً تمام این تحلیل‌های زیر گروهی و داده‌های عمیق را انجام نمی‌دادید».
در اینجا، «گلمن» به فرآیند «فرضیه‌سازی پس از مشخص شدن نتایج» \textenglish{\textbf{(HARKing)}} اشاره می‌کند (در این مورد، به نظر می‌رسد که فرضیه اصلی وانسینک هیچ پشتیبانی پیدا نکرده است، بنابراین داده‌ها به سادگی توسط پست دکتر ترکیه استخراج شد تا ببیند آیا برخی تداعی های قابل قبولی پیدا شد).
بل توصیه می‌کند که محققان می‌توانند با اعلام «فرضیه‌های با انگیزه‌ی واضح، در کنار پیش‌بینی‌های ابطال‌پذیر، قبل از آزمایش، از موارد \textenglish{\textbf{HARKing}} اجتناب کنند».
این امر در بسیاری از زمینه‌ها، از جمله یادگیری ماشینی، از طریق ثبت پیش‌ثبت آزمایش‌ها، از جمله فرضیه‌ها، داده‌ها، تجزیه و تحلیل و طراحی آزمایشی انجام می‌شود.
مخزن \textenglish{\textbf{OpenML}} نمونه خوبی از حرکت به سمت علم باز است.

با استفاده از روش \textenglish{\textbf{p-hacking}}، گلمن به روش بی اعتبار ماساژ داده ها اشاره می‌کند (به عنوان مثال با بازی با اندازه های نمونه) برای ایجاد یک نتیجه به ظاهر آماری مهم در جایی که هیچ کدام واقعا وجود ندارد.\ \textenglish{\textbf{p-hacking}} نیز اعتبار مدل ها را به خطر می اندازد زیرا "فرض اصلی یک آزمون فرضیه آماری را باطل می‌کند: احتمال اینکه یک نتیجه منفرد به دلیل شانس باشد".
\mbox{\textenglish{\textbf{p-hacking}}} می‌تواند ما را به پذیرش نتایج معتبری که صرفاً تصادفی هستند سوق دهد.
\textenglish{\textbf{p-hacking}} به \textenglish{\textbf{HARKing}}، لایروبی داده‌ها و گزارش نتایج بسیار مهم به عنوان شیوه هایی می پیوندد که مدل های نامعتبر را در یادگیری ماشین نیز تولید می‌کنند.
مجموعه داده‌های بزرگی که در یادگیری ماشین استفاده می‌شوند، به‌ویژه در ایجاد نتایج مثبت نادرست هستند (برای تعاریف جنبه‌های اصلی روش علمی به \hyperref[sec:جعبه 1.3]{\mbox{کادر 1.3}} مراجعه کنید).
گلمن پست وبلاگ خود را با بیان این جمله به پایان رساند: «از جمله آخری که رزومه «همیشه پنج مقاله خواهد داشت» آزارم می‌دهد.
وضعیت نهایی تحقیق رزومه نیست.


% ======================================== کادر 3.1
\begin{tcolorbox}[colback=gray!10,colframe=black]

    \phantomsection
    \section*{جعبه 1.3}
    \label{sec:جعبه 1.3}
    \begin{Large}
        \textbf{\mbox{روش علمی}}
    \end{Large}
    \newline

    \begin{description}[leftmargin=0.5cm,style=nextline]
        \item[تکرارپذیری:] نتایج به دست آمده در یک کارآزمایی یا آزمایش زمانی مشابه خواهد بود که در شرایط مشابه تکرار شود، که نیاز به مستندسازی توسط محققین به گونه ای کامل و همچنین شفاف دارد.
        همچنین به عنوان تکرارپذیری و تکرارپذیری شناخته می‌شود.
        \item[قابلیت اطمینان:] معیاری برای قابلیت اطمینان.
        همچنین به عنوان قابلیت اطمینان تست/آزمون مجدد شناخته می شود.
        فردی که چندین بار در آزمون شرکت می‌کند، تا حد زیادی پاسخ های مشابهی می‌دهد.
        سیستمی که چندین بار در شرایط یکسان اجرا می شود، نتایج تا حد زیادی در طول زمان ایجاد می‌کند.
        \item[دقت:] اندازه‌گیری‌ها یا آزمایش‌هایی که نتایجی شبیه به یکدیگر ایجاد می‌کنند.
        \item[صحت:] اندازه گیری خطا بین اندازه گیری های متوسط و مقدار واقعی.
        \item[اعتبار:] میزانی که یک مدل یا اندازه‌گیری ادعا شده دقیقاً آنچه را که ادعا می‌کند منعکس می‌کند.
    \end{description}


\end{tcolorbox}


این رسوایی پایان کار برای «وانسینک» بود.
دانشمندان دیگر شروع به درخواست داده‌های اصلی در مطالعات ناهار مدرسه کردند، اما هیچ کدام یافت نشد.
سپس نام تجاری و کاغذ ناهار مدرسه نیز پس گرفته‌شد.
در سپتامبر 2018، وانسینک پس از تحقیقاتی که در کورنل انجام شد، بازنشسته‌شد و نشان‌داد که او واقعاً مرتکب تخلفات تحقیقاتی، از جمله گزارش نادرست داده‌ها، داده‌های از دست‌رفته، خطاهای آماری و اسناد نامناسب نویسندگی شده‌است.
سال قبل، تحقیقات آن‌ها "خطا" پیدا کرده بود، اما "سوء رفتار" وجود نداشت.

انتقادات از تحقیقات وانسینک در زمان حساسی برای بحران تکرار مطرح شد و سینگال اظهار داشت که او یکی از تراژدی‌های بزرگ آن بحران بود.
وانسینک و آزمایشگاه او، ناشران پرکارِ مطالعات جلب توجه بودند (روشی که اغلب منجر به خطاهای کنترل کیفیت می شود، مانند آنچه در اینجا دیدیم).
بحران تکرارپذیری، البته، بسیار بیشتر از تکرارپذیری است.
این در مورد ماهیت خود روش علمی \hyperref[sec:جعبه 1.3]{\mbox{کادر 1.3}} و معنای تولید نظریه ها، مدل ها و دانشی است که تصویری عینی درست از واقعیت ارائه می‌دهد.
بسیاری از نتایج در روانشناسی، پزشکی و علوم اجتماعی قابل تکرار نیستند (و بنابراین احتمالاً نیز نامعتبر هستند) (به \hyperref[sec:جعبه 2.3]{\mbox{کادر 2.3}} مراجعه کنید).

\newpage

% ======================================== کادر 3.2
\begin{tcolorbox}[colback=gray!10,colframe=black,breakable]

    \phantomsection
    \section*{جعبه 2.3}
    \label{sec:جعبه 2.3}
    \begin{Large}
        \textbf{\mbox{چک لیست تکرارپذیری}}
    \end{Large}
    \newline
    برای همه مدل ها و الگوریتم های ارائه شده، بررسی کنید که آیا شامل موارد زیر است:

    \setstretch{0.8} % Adjust the value as desired
    \begin{itemize}[itemsep=0.2ex]
        \item * شرح واضحی از تنظیمات ریاضی، الگوریتم و/یا مدل.
        \item * توضیح واضح در مورد هر فرضی.
        \item * تجزیه و تحلیل پیچیدگی (زمان، مکان، اندازه نمونه) هر الگوریتم.
        \newline
    \end{itemize}

    برای هر ادعای نظری، بررسی کنید که آیا شامل موارد زیر است:
    \setstretch{0.8} % Adjust the value as desired
    \begin{itemize}[itemsep=0.2ex]
        \item * بیان واضح ادعا.
        \item * اثبات کامل ادعا.
        \newline
    \end{itemize}

    برای همه مجموعه داده‌های مورد استفاده، بررسی کنید که آیا شامل موارد زیر است:
    \setstretch{1} % Adjust the value as desired
    \begin{itemize}[itemsep=0.2ex]
        \item * آمار مربوطه، مانند تعداد نمونه.
        \item * جزئیات تقسیم قطار/ اعتبارسنجی/آزمایش.
        \item * توضیحی در مورد هر داده ای که حذف شده است، و تمام مراحل قبل از پردازش.
        \item * پیوندی به نسخه قابل دانلود مجموعه داده یا محیط شبیه سازی.
        \item * برای داده های جدید جمع‌آوری شده، شرح کاملی از فرآیند جمع‌آوری داده‌ها، مانند دستورالعمل‌ها به حاشیه نویس‌ها و روش‌های کنترل کیفیت.
        \newline
    \end{itemize}

    برای همه کدهای مشترک مرتبط با این کار، بررسی کنید که آیا شامل موارد زیر است:
    \setstretch{1} % Adjust the value as desired
    \begin{itemize}[itemsep=0.2ex]
        \item * تعیین وابستگی ها.
        \item * کد آموزشی.
        \item * کد ارزیابی.
        \item * مدل(های) (از قبل) آموزش دیده.
        \item * فایل \textenglish{\textbf{README}} شامل جدولی از نتایج است که با دستور دقیق اجرا برای تولید آن نتایج همراه است.
        \newline
    \end{itemize}


    برای همه‌ی نتایج آزمایشی گزارش شده، بررسی کنید که آیا شامل موارد زیر است:
    \setstretch{1} % Adjust the value as desired
    \begin{itemize}[itemsep=0.2ex]
        \item * محدوده فراپارامترهای در نظر گرفته شده، روش انتخاب بهترین پیکربندی هایپرپارامتر، و مشخصات تمام پارامترهای هایپر مورد استفاده برای تولید نتایج.
        \item * تعداد دقیق دوره های آموزشی و ارزیابی.
        \item * تعریف روشنی از معیار یا آمار خاص مورد استفاده برای گزارش نتایج.
        \item * شرح نتایج با گرایش مرکزی (مثلاً میانگین) و تنوع (مثلاً نوارهای خطا).
        \item * میانگین زمان اجرا برای هر نتیجه، یا هزینه انرژی تخمینی.
        \item * شرح زیرساخت محاسباتی مورد استفاده.
        \newline
    \end{itemize}

    منبع: \textenglish{\textbf{Pineau, Joelle}}.
    چک لیست تکرارپذیری یادگیری ماشین (نسخه 2.0، 7 آوریل 2020).
    \\
    \href{https://www.cs.mcgill.ca/~jpineau/ReproducibilityChecklist-v2.0.pdf}{\textenglish{\textbf{www.cs.mcgill.ca/~jpineau/ReproducibilityChecklist-v2.0.pdf}}}
\end{tcolorbox}

بنابراین، بحران تکرارپذیری به روش‌شناسی ضعیف و همچنین فقدان اعتبار اشاره دارد: نتایج حاصل از روش‌شناسی غیراخلاقی منجر به مدل‌هایی می‌شود که معتبر نیستند (و بنابراین اطلاعات قابل اعتمادی در مورد دنیای واقعی به ما نمی‌دهند).
این کتاب چندین مطالعه موردی را مورد بحث قرار می‌دهد که این می‌تواند منجر به آسیب‌های قابل توجهی شود (از جمله محکومیت‌های نادرست، بازداشت‌های غیرضروری، آزادی افراد خطرناک در شرایط نامناسب، و حتی نسل‌کشی، پاک‌سازی قومی، و خشونت سیاسی).
البته اخلاق تحقیق تنها زمانی به نتایج معتبر و مستحکمی منجر خواهدشد که خود، حوزه‌ی فرهنگی ایجاد‌کند که به اخلاق علمی و دقت روش‌شناختی اهمیت می‌دهد.
روانشناسان دریافته‌اند که پرورش فرهنگ تحقیق اخلاقی نه تنها به اطمینان از تکرارپذیری بلکه اعتبار واقعی نتایج منتشر‌شده کمک می‌کند.
روش‌شناسی خوب همچنین باعث ایجاد اعتماد در بین محققان می‌شود.
همانطور که «هایل» خاطرنشان می‌کند، "هیچ دانشمندی نمی‌تواند از هر مقاله‌ای که می‌خواند، نتایج را بازتولید کند" و تعداد بسیار کمی از مقالات منتشر‌شده حتی یک تلاش برای بازتولید مشاهده خواهند‌کرد.
بقیه را ما اعتماد می‌کنیم.

جنچوغلو \textenglish{\textbf{(Gencoglu)}} به این نکته اشاره می‌کند که ما به بسیاری از مطالعات موردی که در ادامه می‌آیند باز خواهیم گشت: یک فرهنگ تحقیقاتی دقیق در یادگیری‌ماشین باید «به نیازهای انسان و روانشناسی به شیوه‌ای واقع‌بینانه رسیدگی کند».
برای انجام این کار، «متخصصان سطح بالا باید از ابتدا در تیم‌های مطالعاتی گنجانده شوند»، به‌ویژه به‌عنوان تلاش‌های یادگیری‌ماشین در زمینه‌هایی که مدت‌هاست تخصص خود را توسعه داده‌اند (شواهد پزشکی قانونی، ارزیابی خطر در جرم‌شناسی، بیومتریک، اثرات رسانه‌ها، و قوانین آزادی بیان، از جمله).

در پایان، هیچ ترفند ساده‌ای وجود ندارد که اطمینان حاصل کند که تحقیقات به ما دانش معتبر می‌دهد و تصویری دقیق و مفید از واقعیت ارائه می‌دهد که ما در تلاش برای درک و مدل سازی آن هستیم (همانطور که هیچ ترفند ساده ای برای یادگیری نحوه انجام آن وجود ندارد).
سالم غذا بخورید، تصمیم بگیرید که چه محتوایی باید در رسانه‌های اجتماعی ممنوع شود، یا اینکه در یک محاکمه جنایی گناه و بی‌گناهی را تعیین کنید.
در زمینه‌ای جوان و به سرعت در حال رشد مانند یادگیری ماشین، فرهنگی لازم است که روش‌های قوی و اعتبار مدل‌ها را ارج می‌نهد (فرهنگی که در مورد تولید دانشی که نیازهای مردم را برآورده می‌کند و در آزمون زمان مقاومت می‌کند تأمل می‌کند).
\\
در ادامه‌، یک تفسیر از رفتار سودگرا را خدمت شما ارائه میدهیم.

\newpage

{\setstretch{0.5}
\phantomsection
\section*{تفسیر}
\label{sec:تفسیر}
\addcontentsline{toc}{section}{تفسیر}{\protect\numberline{}}


\phantomsection
\subsection*{اخلاق سودگرا}
\label{subsec:اخلاق سودگرا}
\addcontentsline{toc}{subsection}{اخلاق سودگرا}{\protect\numberline{}}
\textbf{توسط «پیتر سینگر» و «ییپ فای تسه»}
\\\\
}
از دیدگاه فایده‌گرا، رفتار «وانسینک» غیراخلاقی است، زیرا خطر عواقب منفی بیشتری را نسبت به منافع بالقوه ایجاد می‌کند.
حوزه علمی را تصور کنید که اکثریت یا حتی بخش قابل توجهی از پزشکان از نظر فکری صادق نیستند.
تحقیق در آن زمینه قابل استناد نیست.

به نظر می‌رسد «وانسینک» یک دستور کار در پشت تحقیقات خود دارد: او می‌خواست مردم به روش خاصی (سالم، همانطور که او معتقد بود) غذا بخورند.
این ممکن است دلیلی باشد که او فقط از نتایجی حمایت می‌کند که نظرات او را تأیید می‌کند.
اینکه آرزو کنیم مردم به روش خاصی غذا بخورند، البته لزوماً بد نیست.
و ممکن است، شاید به احتمال زیاد، نیت او خیر بوده باشد.
اما نیت خوب، ناصادق بودن را توجیه نمی‌کند (به عبارت دیگر، حسن فاعلی داشته، ولی حسن فعلی نداشته).

داشتن نیت خیر به خودی خود برای اخلاقی عمل‌کردن کافی نیست.
همچنین باید به روشی مبتنی بر شواهد، تجربی و نظری درست عمل کرد.
یک فرد با نیت خوب، با یافتن شواهد یا دلایلی علیه دستور کار خود، نیاز به ارزیابی مجدد دارد، و شاید، اگر دلایل به اندازه کافی قوی باشد، دستور کار خود را رد کند.

نادیده گرفتن شواهد و استدلال‌ها علیه دستور کار خود، ممکن است نیت خوب را به خیال‌پردازی‌های خودفریبی تبدیل کند.
همچنین می‌تواند آسیب جدی، احتمالاً در مقیاس وسیع، ایجاد کند.
در مورد وانسینک، او بسیار بیشتر از شغل خود و شهرت رشته و موسسه خود به خطر انداخت.
او همچنین خطر ارائه توصیه های ناآگاهانه در مورد عادات غذایی را داشت و در نتیجه به کسانی که از توصیه های او پیروی می‌کردند آسیب می‌رساند.

صداقت فکری تنها شرط اخلاقی نیست.
محققان، به‌ویژه آن‌هایی که روی پروژه‌هایی کار می‌کنند که به طور بالقوه می‌توانند به زندگی موجودات ذی‌شعور آسیب بزنند، از نظر اخلاقی مسئول تأثیرات قابل پیش‌بینی تحقیقات خود هستند.
به عنوان مثال، تأثیر تحقیق در زیست شناسی می‌تواند قابل‌توجه باشد، زیرا اغلب پیامدهای عمده‌ای بر بسیاری از انسان‌ها و حیوانات دارد.

نگرانی اخیر در مورد استفاده از فناوری \textenglish{\textbf{(CRISPR)}} برای قادر ساختن تروریست‌ها به اصلاح ویروس‌ها برای اهداف حمله، تنها نمونه‌ای از این است که چگونه بیوتکنولوژی می‌تواند تأثیرات عظیمی ایجاد کند.

علم داده، حداقل به اندازه‌ی زیست‌شناسی تأثیر مورد انتظار دارد.
مهم است که محققان قبل از انتشار، یا حتی بهتر از آن، حتی قبل از انجام تحقیقات خود در زمینه‌های خاص، در مورد پیامدهای اخلاقی کار خود به دقت فکر کنند.



