%! Author = zamoosh
%! Date = 6/8/23


\chapter{مدل‌های ماشین در دادگاه}
\label{ch:مدل‌های ماشین در دادگاه}
\phantomsection

\begin{quote}
    نتیجه‌گیری‌های علمی در معرض تجدید‌نظر دائمی هستند.
    از سوی دیگر، قانون باید اختلافات را به سرعت و در نهایت حل کند.
    پروژه علمی با بررسی گسترده و گسترده انبوهی از فرضیه‌ها پیش می‌رود، زیرا فرضیه‌هایی که نادرست هستند، در نهایت نشان‌داده‌می‌شوند که چنین هستند، و این خود یک پیشرفت است.
    با این حال، حدس‌هایی که احتمالاً اشتباه هستند، در پروژه دستیابی به یک قضاوت حقوقی سریع، نهایی و الزام آور (اغلب دارای پیامدهای بزرگ) در مورد مجموعه خاصی از رویدادهای گذشته کاربرد چندانی ندارند.
    \\\\
    \textbf{(قاضی «بلکمون»، «دابرت» علیه داروسازی «مرل داو»، دادگاه عالی ایالات متحده، 1993)}
\end{quote}


\newpage


{\setstretch{0.5}
\phantomsection
\section*{محاکمه‌های شفاهی نیکلاس هیلاری}
\label{sec:محاکمه‌های شفاهی نیکلاس هیلاری}
\addcontentsline{toc}{section}{محاکمه‌های شفاهی نیکلاس هیلاری}{\protect\numberline{}}
در «24 اکتبر 2011»، یک قتل عجیب و وحشتناک در پوتسدام، نیویورک (شهری کوچک در کنار رودخانه سنت لارنس و بسیار نزدیک به مرز با استان انتاریو کانادا) اتفاق افتاد. پسر 12 ساله‌ای به نام «گرت فیلیپس» حوالی ساعت 5 بعدازظهر، مدت کوتاهی پس از بازگشت از مدرسه به خانه، در اتاق خوابش خفه‌شد. همسایه‌ها غوغایی را شنیدند و با 911 (پلیس آمریکا) تماس گرفتند. پرده‌های پنجره اتاق خواب طبقه دوم «گرت» به سمت بیرون خم شده‌بود و باعث‌شد تا بازرسان شک کنند که قاتل از آن طرف بیرون‌پریده و فرار‌کرده است.
}

پلیس به سرعت در مورد «اورال نیکلاس (یا به اختصار نیک) هیلاری» به عنوان مظنون اصلی خود در این پرونده و به دلایلی که کاملاً واهی به نظر می رسد، مظنون شد.
هیلاری مربی فوتبال تیم مردان در «دانشگاه کلارکسون» بود.
هیلاری، بسیار موفق و محبوب بود به طوری که تقریباً همه در پوتسدام می‌دانستند او کیست.
او، اخیراً با «تاندی سایرس» (مادر گرت) در رابطه‌ی عاشقانه بوده.
او همچنین یکی از معدود آمریکایی‌های آفریقایی‌تبار ساکن در پوتسدام بود و رابطه‌ی او با «تندی موجی»، شوک در جامعه ایجاد کرده‌بود.
آن‌ها اخیراً از هم جدا شده‌بودند زیرا دو پسر «تاندی» با نیک کنار‌نمی‌آمدند که باعث ایجاد مشکلاتی در خانواده شده‌بود.
علاوه بر این، نیک و تاندی هر دو در زمان شروع رابطه با افراد دیگر در ارتباط بودند.
در آن زمان، تاندی در حال دیدن جان جونز (یک کلانتر در پوتسدام) بود.
«جان جونز» از اینکه نیک عاملی در جدایی او از تاندی بود بسیار ناراحت بود و به خانه نیک رفت تا با او مقابله کند و احتمالاً او را تهدید کند.
آشکارا تنش‌هایی در جامعه وجود داشت و این احساس عمومی بود که جان نه تنها به خاطر از دست دادن دوست‌دخترش ناامید شده‌بود، بلکه به رقیبی که یک آمریکاییِ آفریقایی‌تبار و بسیار موفق بود شکست خورد.

پلیس خیلی سریع هیلاری را به عنوان یک مظنون دستگیر کرد، چیزی که به نظر می‌رسد یک مورد ناشی از خصومت‌های نژادی و شخصی است.
او چندین ساعت بازداشت و مورد بازجویی و حتی بازرسی قرار‌گرفت تا ببینند آیا جراحاتی مطابق با پرش از پنجره طبقه دوم دارد یا خیر!
او به هیچ‌وجه مقصر نبود.
بدون هیچ مدرکی دالِ بر ارتباط او با قتل گرت، او آزاد شد و بعداً یک شکایت حقوق‌مدنی علیه پلیس تنظیم کرد.

این قضیه، تازه آغاز مشکلات حقوقی هیلاری بود؛ نه پایان آن!
وکیل مدافع در برابر دعوی حقوق‌مدنی، راهبردی را برای اثبات اینکه نیک در واقع مرتکب قتل شده است، ایجاد کرد!
(و او از شهادت هیلاری در هنگام تسلیم شدن علیه خود استفاده کرد).
با وجود شواهد بسیار متزلزل و کم احتمال، دادستان منطقه‌ی «مری راین» در «12 می 2014» کیفرخواستی را برای قتل درجه‌ی دوم علیه هیلاری به دست‌آورد!
این کیفرخواست، در «اکتبر 2014» به دلیل سوء‌رفتار دادستانی از جانب راین \textenglish{\textbf{(Rain)}} ردشد.
«رین» در «2 فوریه 2015»، دومین هیئت منصفه را تشکیل‌داد و کیفرخواست دیگری را برای قتل علیه هیلاری دریافت‌کرد.

این قضیه، تازه آغاز مشکلات حقوقی هیلاری بود؛ نه پایان آن!
وکیل مدافع در برابر دعوی حقوق‌مدنی، راهبردی را برای اثبات اینکه نیک در واقع مرتکب قتل شده است، ایجاد کرد!
(و او از شهادت هیلاری در هنگام تسلیم شدن علیه خود استفاده کرد).
با وجود شواهد بسیار متزلزل و کم احتمال، دادستان منطقه‌ی «مری راین» در «12 می 2014» کیفرخواستی را برای قتل درجه‌ی دوم علیه هیلاری به دست‌آورد!
این کیفرخواست، در «اکتبر 2014» به دلیل سوء‌رفتار دادستانی از جانب راین \textenglish{\textbf{(Rain)}} ردشد.
«رین» در «2 فوریه 2015»، دومین هیئت منصفه را تشکیل‌داد و کیفرخواست دیگری را برای قتل علیه هیلاری دریافت‌کرد.

جامعه در آشفتگی بود زیرا پرونده بدون هیچ راه‌حل روشنی به طول انجامید.
شایعات گسترده‌ای مبنی بر وجود شواهد قوی \textenglish{\textbf{DNA}} علیه هیلاری و سرکوب آن به دلیل "فنی" وجود‌داشت.

\newpage

{\setstretch{0.5}
\phantomsection
\section*{خطرناک ترین دادستان نیویورک}
\label{sec:خطرناک ترین دادستان نیویورک}
\addcontentsline{toc}{section}{خطرناک ترین دادستان نیویورک}{\protect\numberline{}}
آن شواهد \textenglish{\textbf{DNA}} در فضای‌نادری از خصومت‌نژادی علیه هیلاری و شواهد روشنی از سوء‌رفتار دادستانی از سوی \textenglish{\textbf{«DA Mary Rain»}} جمع‌آوری و تفسیر شد. «راین» در ابتدا در سکویی برای حل قتل گرت شرکت‌کرد و به انتقاداتی مبنی بر استفاده او از قتل برای منافع سیاسی منجر‌شد (او اغلب در کنار تاندی سایرس در رویدادهای مبارزات انتخاباتی ظاهر می شد). «راین» به سرعت از موقعیت جدید خود برای آزار و اذیت مقاماتی استفاده کرد که در زمانی که او یک مدافع عمومی بود او را به دلیل بی کفایتی اخراج کرده‌بودند. او در سال 2017 دفتر را زیر ابری از سوء ظن ترک‌کرد بدون اینکه به دنبال انتخاب مجدد باشد. او بعداً به مدت دو سال از وکالت تعلیق‌شد (یک اتفاق بسیار نادر و گواهی بر شدت و تداوم سوء رفتار او به عنوان دادستان).
}

در واقع، رفتار نادرست «راین» در دوران تصدی او به عنوان یک \textenglish{\textbf{DA}} بدنام بود.
در پرونده علیه هیلاری، چندین اقدام غیرقانونی برای افشای اطلاعات دربر‌می‌گرفت.
او این واقعیت را که شاهدی گزارش داده بود جان جونز را در حال ورود به آپارتمان گرت در نزدیکی زمانی که او کشته‌شده‌بود، ردکرد.
«تاندی سایرس» در ژانویه 2011 از «جونز» شکایت کرده بود و اظهار داشت که «جونز» به گونه‌ای عمل می‌کند که باعث ترس او برای امنیت خود و فرزندانش می‌شود، از جمله اینکه جونز بدون اعلام قبلی و بدون دعوت به آپارتمانش می‌رود، علی‌رغم اینکه مکرراً به او گفته‌شده‌بود که نکند.

اگرچه او یک مظنون معقول در این پرونده بود، «جونز» به «راین» حقایقی داد که او به راحتی آن را پذیرفت.
«راین» با این ادعا که اظهارات شاهد «با نظریه [دادستان] پرونده مطابقت ندارد» و بنابراین دلیلی برای افشای آن به دفاع وجود ندارد، سرکوب شواهد را توجیه کرد.
این احتمالاً مقصرترین و غیراخلاقی‌ترین دلیل ممکن برای ناتوانی دادستان در افشای شواهد تبرئه‌کننده است (به‌ویژه زمانی که مظنون مورد بحث یک کلانتر محلی است که فعالانه در تحقیقات قتل شرکت داشته است).

اولین پرونده هیئت منصفه علیه «هیلاری» به دلیل رفتار غیراخلاقی «راین»، رد شد.
«قاضی ریچاردز» حکم داد که راین این روند را لکه‌دار کرده‌است، از جمله با نشان‌دادن دختر 17 ساله «هیلاری» برای افشای ارتباطاتی که توسط امتیاز وکیل-موکل محافظت می‌شود.
در همان زمان، \textenglish{\textbf{FBI}} در حال تحقیق از راین برای تماس با زندانیان بدون رضایت وکیلشان بود تا آن‌ها را متقاعد‌کند که علیه سایر زندانیان شهادت‌دهند.
خبرچینان زندان که برای شهادت تحت فشار قرار گرفته‌اند در حالی که از حقشان برای داشتن وکیل محروم شده‌اند، شواهد بسیار غیرقابل اعتمادی ارائه‌می‌کنند، و نشان‌داده‌‌شده‌است که این عامل مهمی در محکومیت‌های نادرست است.
\newline
\newline


{\setstretch{0.5}
\phantomsection
\section*{شواهد DNA}
\label{sec:شواهد DNA}
\addcontentsline{toc}{section}{شواهد DNA}{\protect\numberline{}}
محاکمه‌ی هیلاری برای قتل «گرت فیلیپس» تنها در مقابل یک قاضی در شهرستان سنت لارنس برگزار‌شد. هیلاری ممکن است یک محاکمه روی نیمکت را انتخاب کرده باشد، زیرا فکر‌می‌کرد (احتمالاً به درستی) ممکن است هیئت منصفه محلی، با او منصف نباشد. در محاکمه شواهد فیزیکی کمی در دسترس‌بود. چهار اثر انگشت نهفته روی پنجره طبقه دوم و اطراف آن پیدا‌شد که گمان‌می‌رود عامل جنایت از آنجا فرار کرده است. اثر انگشت‌ها متعلق به هیلاری نبودند و هرگز با کسی که به این پرونده مرتبط است یا کسی در پایگاه داده ایالت نیویورک \textenglish{\textbf{SAFIS}} مطابقت نداشت.
}

همچنین مقادیر کمی از شواهد \textenglish{\textbf{DNA}} وجود داشت که در نهایت به اثبات این پرونده منجر‌شد.
مشخصات \textenglish{\textbf{DNA}} از تراشیدن ناخن‌های دست جمع‌آوری‌شده در کالبد شکافی گرت ایجاد‌شد.
محققان نظریه ای را ارائه‌کردند که ممکن است گرت قبل از مرگ با مهاجم خود مبارزه‌کرده باشد.
با این حال، \textenglish{\textbf{DNA}} را فقط می‌توان در مقادیر کمی بازیابی کرد، که نشان‌می‌دهد ممکن است دستکاری قابل‌توجهی در \textenglish{\textbf{DNA}} وجود داشته‌باشد، یا ممکن است بخشی از پس‌زمینه یا آلودگی به واسطه محقق بوده‌باشد.
از آنجایی که \textenglish{\textbf{DNA}} تعداد کپی پایینی داشت، کمتر از میزان توصیه‌شده برای تجزیه و تحلیل قرار‌گرفت.
این به نوبه‌ی خود، تفسیر مشخصات \textenglish{\textbf{DNA}} را بسیار دشوارتر می‌کند و ارزش اثباتی شواهد را زیر سؤال می‌برد.
اثبات شده‌است که تفسیر پروفایل‌های \textenglish{\textbf{DNA}} تخریب‌شده، کم کپی و مختلط برای دانشمندان پزشکی قانونی و دادگاه‌ها، یک چالش است.
با توجه به پیچیدگی محاسبه احتمال اینکه مشخصات \textenglish{\textbf{DNA}} یک فرد معین در نمونه گرفته‌شده از صحنه جرم گنجانده‌شود، الگوریتم‌های متعددی برای تخمین احتمالات گنجاندن و نسبت‌های احتمال ایجاد شده‌است.
روش‌های سنتی تخمین احتمالات برای نمونه‌های کم‌کپی و مخلوط، احتمال یکسانی را به همه ژنوتیپ‌ها اختصاص می‌دهند که ارزش اثباتی این شواهد را محدود می‌کند.
الگوریتم‌های تفسیر \textenglish{\textbf{DNA}} وزن‌های آماری را به ژنوتیپ‌های مختلف اختصاص می‌دهند (از جمله احتمال اینکه آلل‌های خاصی ممکن است «از بین بروند» و در نمونه ظاهر نشوند)، یا اینکه یک آرتیفکت ممکن است به‌عنوان یک آلل ظاهر شود و در نتیجه زمانی که در واقع گنجانده نشده‌است، «افت کند».
نمایه‌ی \textenglish{\textbf{DNA}} که از زیر ناخن‌های گرت ایجاد شد، یک نیمرخ جزئی بود، به این معنی که چندین آلل از بین‌رفته‌بودند و در الکتروفروگرام قابل تشخیص نبودند.
بنابراین، این الگوریتم‌ها کار بسیار بهتری را برای تخمین اینکه آیا یک فرد معین در یک نمونه پیچیده گرفته‌شده از صحنه‌ی جرم گنجانده‌شده‌است یا خیر، انجام می‌دهند.
دو تا از محبوب‌ترین مدل‌های نرم‌افزار تفسیر مخلوط \textenglish{\textbf{DNA}} تجاری موجود، \textenglish{\textbf{TrueAllele}} و \textenglish{\textbf{STRmix}} هستند.
هر دو در مورد هیلاری مورد استفاده قرار‌گرفتند و هر دو به نتایج متفاوتی رسیدند!
که آیا \textenglish{\textbf{DNA}} او در نمایه‌ی ایجاد شده از خراش‌دادن ناخن گرت گنجانده شده است یا خیر.

«جان باکلتون»، متخصص ژنتیک پزشکی قانونی پیشرو که نقشی کلیدی در توسعه \textenglish{\textbf{STRmix}} داشت، بیان می‌کند که الگوریتم‌های مخلوط \textenglish{\textbf{DNA}} می‌توانند نمونه‌های پیچیده \textenglish{\textbf{DNA}} را با سرعت و دقت بیشتری تجزیه و تحلیل کنند.
نرم‌افزار تفسیر \textenglish{\textbf{DNA}} به طور کلی با استفاده از روش‌های زنجیره مارکوف مونت کارلو \textenglish{\textbf{(MCMC)}} برای حل مخلوط‌ها و توسعه‌ی احتمالات مشروط گنجاندن کار می‌کند.
روش‌های \textenglish{\textbf{(MCMC)}} مدت‌هاست که در مدل‌های یادگیری ماشین مورد استفاده قرار می‌گیرند (زیرا مدت‌هاست که در بسیاری از زمینه‌ها از جمله فیزیک، اقتصاد سنجی و علوم کامپیوتر برای حل مسائل با ابعاد بالا مورد استفاده قرار گرفته‌اند).
این مهم است که دانش دامنه سطح بالا را در مدل \textenglish{\textbf{(MCMC)}} برای ایجاد فهرست مناسبی از فرضیه‌های نامزد وارد کنید.

فرضیه‌های نامزد در این مورد شامل اینکه آیا \textenglish{\textbf{DNA}} هیلاری در نمونه گنجانده شده‌است یا خیر.
آیا او به عنوان مشارکت‌کننده در نمونه حذف شده‌است یا خیر.
آیا نمونه \textenglish{\textbf{DNA}} از طریق آلودگی پس‌زمینه زیر ناخن‌های گرت قرار گرفته‌است و دلیل اثباتی در قتل او نیست (معمولی نیست که به دلیل آلودگی ناشی از فعالیت‌های روزمره، \textenglish{\textbf{DNA}} افراد دیگر را در زیر ناخن‌هایمان به مقدار کمی پیدا کنیم).
و اینکه آیا \textenglish{\textbf{DNA}} پس از جنایت از طریق نوعی آلودگی با واسطه بازپرس معرفی شده‌است یا خیر.

برای تفسیر \textenglish{\textbf{DNA}}، این دانش حوزه همچنین شامل قوانینی برای یافتن شواهد قابل‌قبول در دادگاه (\hyperref[sec:کادر 1.4]{کادر 1.4}) و همچنین دستورالعمل‌های \textenglish{\textbf{SWGDAM}} برای تأیید سیستم‌های ژنوتیپ احتمالی است.
قبل از استفاده، یک تحلیلگر \textenglish{\textbf{DNA}} پزشکی قانونی باید پیک‌ها را تفسیر کند، افت و ریزش را تخمین بزند، و در واقع آلل‌ها را فراخوانی کند.
نرم‌افزار ابتدا باید به‌صورت داخلی توسط آزمایشگاه تحت شرایطی مشابه شرایط نمونه صحنه‌ی جرم تأیید‌شود.
این کنترل‌های کیفی اولیه در مورد هیلاری استفاده نشدند!
«قاضی فلیکس کاتنا» یک جلسه استماع (\textenglish{\textbf{Frye}}، اسم آزمونی در دادگاه‌های ایالات متحده) فرای برگزار‌کرد تا مشخص کند که آیا شواهد \textenglish{\textbf{STRmix}} با توجه به اینکه از \textenglish{\textbf{DNA}} با تعداد کم کپی گرفته‌شده بود و فقط از یک نمایه \textenglish{\textbf{DNA}} جزئی تشکیل‌شده‌بود، قابل‌قبول است یا خیر.
آزمون‌فرای یکی از دو استاندارد اصلی است که دادگاه‌ها در ایالات متحده برای تعیین اینکه آیا شواهد کارشناسی در دادگاه قابل‌قبول هستند یا خیر (\hyperref[sec:کادر 1.4]{کادر 1.4}) استفاده می‌کنند.
ماهیت آزمون فرای این است که اگر علمی که نظر مبتنی بر آن است به طور کلی در آن جامعه علمی قابل اعتماد تلقی‌شود، شواهد کارشناسی پذیرفته می‌شود.
آزمون کلیدی دیگر پذیرش، که در دابرت ارائه شده‌است، نه تنها به این می‌پردازد که آیا تکنیک یا نظریه مورد قبول است یا خیر، بلکه به این موضوع می‌پردازد که آیا می‌توان آن را آزمایش کرد و آزمایش شده‌است، آیا میزان خطای شناخته‌شده‌ای برای رویه وجود دارد، آیا کنترل کیفیت وجود دارد یا خیر.
و سایر استانداردهای حاکم بر رویه، و اینکه آیا مورد بررسی همتایان قرار‌گرفته‌است (\hyperref[sec:کادر 2.4]{کادر 2.4}).





% ======================================== کادر 4.1
\begin{tcolorbox}[colback=gray!10,colframe=black,breakable]

    \phantomsection
    \section*{کادر 1.4}
    \label{sec:کادر 1.4}
    \begin{Large}
        \textbf{\mbox{قواعد اولیه مدرک}}
    \end{Large}
    \newline
    قواعد شواهد حجیم هستند و هر حوزه قضایی تغییرات خاص خود را خواهد داشت.
    ادله به طور کلی در دادگاه قابل پذیرش است اگر موارد زیر باشد:

    \begin{description}[leftmargin=0.5cm,style=nextline]
        \item[مربوط:] شواهد در صورتی مرتبط هستند که به دادگاه کمک‌کنند تا به سؤالی که مورد اختلاف است یا تمایل به اثبات یا رد واقعیتی مهم دارد، پاسخ دهد.
        ارزش اثباتی شواهد به میزان تمایل یک مدرک به اثبات یا رد واقعیت مورد اختلاف اشاره دارد.
        \item[قابل اعتماد:] شواهدی که غیرقابل اعتماد هستند، یا می‌خواهند یک داور حقیقت را گمراه‌کنند (هیئت منصفه یا قاضی که به تنهایی نشسته است).
        شواهدی که از دانش دست اول به دست می‌آیند، یا مطابق با روش‌های کنترل کیفیت یا توسط یک مرکز آزمایشگاهی معتبر جمع‌آوری شده‌اند، اغلب قابل اعتمادتر در نظر گرفته می‌شوند.
        وزنی که باید توسط محاکم کننده واقعیت به شواهد داده‌شود، اغلب به میزان قابل اعتماد بودن شواهد بستگی دارد.
        \item[لازمه:] دلیل برای اثبات یا نفی یک امر یا موضوع مورد اختلاف ضروری است.
        اگر شواهد و مدارک دیگری را کپی کنند، غیرضروری خواهند‌بود.
        از سوی دیگر، اگر هیچ راه دیگری برای طرفین وجود نداشته‌باشد که آن شواهد را در دادگاه ارائه کند، ممکن است شواهد لازم باشد، که دادگاه هنگام ارزیابی قابل اعتماد بودن و منصفانه بودن آن مدارک مورد توجه قرار خواهد گرفت.
        شواهد جمع‌آوری شده توسط سیستم یادگیری ماشین در غیاب هر اپراتور انسانی ممکن است ضروری باشد.
        \item[مستثنی نشده:] بسیاری از قواعد استثنایی وجود دارد که ممکن است دادگاه را ملزم به حذف شواهدی کند که در غیر این صورت قابل پذیرش هستند.
        به عنوان مثال، قواعد شنیده‌ها، یا ارتباطات ممتاز، می‌تواند منجر به حذف شواهد قابل اعتماد و اثباتی شود.
        \item[منصفانه:] پذیرش شواهد اغلب منصفانه تلقی می‌شود اگر ارزش اثباتی آن بر پیش داوری ناعادلانه ای که ممکن است برای یک طرف ایجاد کند، بیشتر باشد.
        بسیاری از حوزه‌های قضایی نیز قوانین اساسی خود را خواهند داشت که بر تحقیقات پلیس نظارت می‌کند و به دادگاه‌ها کمک می‌کند تا تعیین کنند که چه زمانی مدارک به شیوه‌ای غیرمنطقی جمع‌آوری شده‌اند، مانند متمم چهارم قانون اساسی ایالات متحده.
    \end{description}
\end{tcolorbox}

این مورد، شواهد DNA پزشکی قانونی با استانداردهای «فرای» یا «دابرت» مطابقت نداشت.
نتایج اولیه از شواهد مخلوط DNA توسط TrueAllele مورد تجزیه و تحلیل قرار گرفت، اما به دلیل کیفیت ضعیف مشخصات DNA، هیچ نتیجه‌ای بدست نیامد: هیلاری نه می‌تواند شامل شود و نه از نمونه حذف می‌شود.
TrueAllele در آن زمان به این نتیجه رسید که آنها «هیچ پشتوانه آماری» پیدا نکردند که هیلاری در ترکیب DNA گرفته‌شده از زیر ناخن‌های «گرت» مشارکت داشته است.
TrueAllele بیان می‌کند که بیش از 100 مورد علاقه را در این پرونده بررسی کرده است، "و نشان داد که هیلاری به شواهد DNA در این پرونده مرتبط نیست."



% ======================================== کادر 4.2
\begin{tcolorbox}[colback=gray!10,colframe=black,breakable]

    \phantomsection
    \section*{کادر 2.4}
    \label{sec:کادر 2.4}
    \begin{Large}
        \textbf{\mbox{پذیرش مدارک علمی و کارشناسی در ایالات متحده}}
    \end{Large}
    \newline

    \begin{description}[leftmargin=0.5cm,style=nextline]
        \item[آزمون فرای:] آزمون پذیرش عمومی نیز نامیده می‌شود.
        دادگاه در صورتی که مدارک علمی یا کارشناسی را در جامعه علمی مربوطه پذیرفته‌باشد، می‌پذیرد.
        این آزمون در \textenglish{\mbox{\textbf{Frye v.United States, 293 F.1013 (D.C. Cir.1923)}}}  در موردی تنظیم شده‌است که شواهد چاپ گراف را کنار گذاشته است زیرا به طور کلی به عنوان قابل اعتماد پذیرفته نشده است.
        برخی از حوزه‌های قضایی ایالات متحده از این آزمون استفاده می کنند، مانند واشینگتن، کالیفرنیا، ایلینوی، مینه سوتا، نیویورک، نیوجرسی و پنسیلوانیا.
        اکثر ایالات دیگر آزمون دابرت را پذیرفته‌اند.

        \item[آزمون دابرت:] این استاندارد از شواهد کارشناسی در دابرت علیه «مرل داو» داروسازی، ایالات متحده، شماره 579 (1993) تنظیم شده است و قانون 702 قوانین فدرال شواهد را به این معنا تفسیر می‌کند که قضات باید یک وظیفه نگهبانی برای اطمینان از اینکه علمی و شواهد کارشناسی مرتبط و قابل اعتماد است.
        دانش علمی، دانشی است که بر اساس روش علمی گردآوری شده باشد و این بستگی به خیلی بیشتر از مقبولیت عمومی دارد.
        دادگاه همچنین می‌تواند بررسی کند که آیا روش‌ها آزمایش و تأیید شده‌اند، آیا میزان خطای شناخته‌شده‌ای وجود دارد، آیا این روش توسط همتا بررسی شده است، آیا به‌طور خاص برای پرونده حاضر تولید شده است یا اینکه مورد قبول و استفاده قرار گرفته است.
    \end{description}

    \textbf{قانون 702. شهادت شهود خبره:} شاهدی که از نظر دانش، مهارت، تجربه، آموزش یا تحصیل صلاحیت کارشناس را داشته باشد، می‌تواند به صورت نظر یا در موارد دیگر شهادت دهد:

    \begin{description}[leftmargin=0.5cm,style=nextline]
        \item[الف)] دانش علمی، فنی، یا سایر دانش‌های تخصصی کارشناس، به آزموده حقیقت کمک می‌کند تا شواهد را درک کند یا واقعیت مورد بحث را تعیین کند.
        \item[ب)] شهادت بر اساس حقایق یا داده‌های کافی است.
        \item[ج)] شهادت محصول اصول و روشهای قابل اعتماد است.
        \item[د)] کارشناس اصول و روش‌ها را به طور قابل اتکایی در مورد واقعیات پرونده اعمال کرده است.
    \end{description}

    \textbf{منبع:} قواعد شواهد فدرال ایالات متحده، اصلاح شده در 17 آوریل 2000، لازم الاجرا در 1 دسامبر 2000.
    و همانطور که در 26 آوریل 2011 اصلاح شد، از 1 دسامبر 2011 لازم الاجرا شد.


\end{tcolorbox}


«ویلیام فیتزپاتریک»، یک DA از «شهرستان اونونداگا»، سپس با «جان باکلتون» تماس گرفت تا ببیند آیا الگوریتم \textenglish{\textbf{DNA}} او، \textenglish{\textbf{STRmix}}، می تواند نتیجه متفاوتی به دست‌آورد.
آن‌ها به \textenglish{\textbf{DNA}} یافت شده در زیر ناخن‌های «گرت» نگاه کردند و مشخصات جزئی آن شامل نیک هیلاری بود.
آزمایشگاه جنایی ایالت نیویورک ابتدا \textenglish{\textbf{DNA}} موجود در نمونه بافت را از طریق \textenglish{\textbf{PCR}} (واکنش زنجیره ای پلیمراز) تکثیر کرد، اما به نظر می‌رسد که واکنش را بیشتر از آنچه توصیه می‌شود در تلاش برای جمع‌آوری مقدار بیشتری از \textenglish{\textbf{DNA}} ردیابی انجام داده است (این باعث افزایش اثرات تصادفی می شود.
و می‌تواند منجر به "افتادن" شود که در آن نویز به عنوان یک آلل در پروفایل \textenglish{\textbf{DNA}} ظاهر می‌شود).
همچنین در روشی که تحلیلگر در ابتدا نمایه جزئی را ارزیابی می‌کرد، سوگیری وجود‌داشت، زیرا او نه تنها می‌دانست که هیلاری مظنون است، بلکه با ارجاع به نمایه خود هیلاری، آلل‌های موجود در نمایه را نام می‌برد.
«لیون» می‌گوید که «یادداشت‌های کاری او نشان می‌دهد که او مشخصات \textenglish{\textbf{DNA}} هیلاری را در حالی که سعی می‌کرد آن را با شواهد مطابقت دهد، بررسی کرده است.» این در تضاد با بهترین شیوه‌ها در علم پزشکی قانونی است که به موجب آن یک تحلیلگر باید نسبت به اینکه مظنونین بالقوه چه کسانی هستند و مشخصات \textenglish{\textbf{DNA}} مظنون چیست، کور باشد تا سوگیری‌های شناختی و زمینه‌ای را در تجزیه و تحلیل خود به حداقل برسانند.
آزمایشگاه یک قطع دلخواه \textenglish{\textbf{50 rfu}} را برای فراخوانی آلل‌ها انتخاب کرد.
به نظر می‌رسید که این پایه جز کمک به حذف برخی از آلل‌های موجود در نمایه هیلاری ندارد، و این امر مستلزم این بود که تحلیلگر به این نتیجه برسد که او در نمونه مشارکت ندارد.
مارک پرلین از \textenglish{\textbf{TrueAllele}} شهادت داد که قله‌هایی درست زیر این آستانه وجود دارد که هیلاری را حذف می کند، و بنابراین نمونه \textenglish{\textbf{DNA}} تبرئه کننده بود.

همانطور که توسط \textenglish{\textbf{STRmix}} توصیه شده است، مطالعات اعتبارسنجی مناسب توسط آزمایشگاه انجام نشد.
زمانی که باکلتون الگوریتم \textenglish{\textbf{STRmix}} را اجرا کرد، فقط الکتروفروگرام به او داده‌شد (که توسط تحلیلگر به شکلی مغرضانه تهیه شده بود) و بنابراین مجبور‌شد «داده‌ها را از «منابع معتبر» مختلف انتخاب و انتخاب کند و پارامترهای ورودی را در برنامه به گونه‌ای وارد کند که او معتقد بود که نظام تحمل خواهد کرد».
آزمایشگاه جنایی ایالت نیویورک مجاز به استفاده از \textenglish{\textbf{STRmix}} بدون مطالعات اعتبار سنجی نبود و همانطور که خود باکلتون توصیه کرده‌بود.
همچنین آزمایشگاه از دستورالعمل‌های خود \textenglish{\textbf{SWGDAM}} پیروی نمی‌کرد، که نیاز به اعتبارسنجی داخلی کامل توسط آزمایشگاه از نمونه‌های پیچیده، کم کپی و نمونه‌های ترکیبی داشت.
همچنین آزمایشگاه از دستورالعمل‌های خود \textenglish{\textbf{SWGDAM}} پیروی نمی‌کرد، که نیاز به اعتبارسنجی داخلی کامل توسط آزمایشگاه از نمونه‌های پیچیده، کم کپی و نمونه‌های ترکیبی داشت.
به این دلایل، «قاضی کاتنا» شواهد \textenglish{\textbf{DNA}} را رد کرد.
این منجر به تبرئه هیلاری شد، زیرا شواهد کمی علیه او وجود داشت.
«مری رین» پس از تبرئه اظهار داشت که علیرغم کمبود شواهدی که این موضوع را تأیید می‌کند (و تعداد مظنونان دیگری که توسط شایعات شهر و خبرنگارانی که این پرونده را پوشش می دادند، \textenglish{\textbf{100\%}} از گناهکار بودن هیلاری مطمئن بودند).
«راین» اظهار داشت که برای هیچ کس دیگری جستجو نمی‌شود، زیرا هیچ کس دیگری نمی‌توانست مرتکب جنایت شود.
\textenglish{\textbf{DA}} جدید، گری پاسکوا، به دنبال سرنخ‌های جدید است، اما قتل گرت فیلیپ حل نشده باقی مانده است.
\newline
\newline


{\setstretch{0.5}
\phantomsection
\section*{تفسیر}
\label{sec:تفسیر}
\addcontentsline{toc}{section}{تفسیر}{\protect\numberline{}}

\phantomsection
\subsection*{اخلاق یهود}
\label{subsec:اخلاق یهود}
\addcontentsline{toc}{subsection}{اخلاق یهود}{\protect\numberline{}}
\textbf{توسط «ساموئل جی لوین»}
\\\\
پرسش‌های اخلاقی حول استفاده و سوء‌استفاده‌ی احتمالی از اشکال شواهد DNA محرمانه در محاکمات جنایی، اگرچه برخاسته از پیشرفت‌های علمی کنونی است، اما نمایانگر اخیر پرسش‌های فلسفی همیشگی است که به دل ماهیت حقوقی و اخلاقی می‌پردازد.
از زمان‌های بسیار قدیم، نظام‌های حقوقی با مفاهیم هنجاری برداشت‌های علمی و فلسفی جدید دست و پنجه نرم کرده‌اند.
با این حال، سرعت پیشرفت تکنولوژی نیاز به در نظر گرفتن کاربردهای عملی موضوعاتی را که تا همین اواخر به نظر می‌رسید در محدوده بحث نظری یا شاید علمی تخیلی باقی می‌ماند، برجسته کرده است.
}

به عنوان یک سیستم فکری که هم قانون و هم الهیات را در‌بر می‌گیرد، اخلاق یهودی مفاهیم به هم پیوسته حقیقت متعالی و واقعیت عملی را بررسی می‌کند.
به عنوان مثال، فیلسوفان حقوقی یهودی به هزاران سال قبل، تنش، اگر نگوییم تناقض، ضمنی در مفاهیم اراده آزاد و جبر، تصدیق کرده‌اند.
قرار‌دادن دانای کل خداوند، از جمله آگاهی از آینده، این پرسش را تشدید می‌کند که آیا مردم باید بر اساس اعمالی که هنوز انجام نداده اند مورد قضاوت قرار گیرند یا خیر؟ این معماها که در منابع متعدد تفکر یهودی به آنها پرداخته شده‌است، گاه با این پذیرش بدیهی حل می‌شود که داوری خداوند ذاتاً عادلانه است، و بنابراین، پاداش و مجازات الهی باید در قلمرو اعمال اراده آزاد انسان صورت گیرد.
شاید تعجب‌آور نباشد که فیلسوفان یهودی تحلیلی از این موضوعات را بر اساس این اصل که قوانین خدا ذاتاً عادلانه هستند، فرض کنند.
با این حال، شاید شگفت‌انگیزتر این باشد که بسیاری از قضات و متفکران حقوقی آمریکایی نیز با کمال میل دکترین اراده آزاد را به عنوان یک نوع ایمان، به جای اینکه نظریه‌های اراده آزاد را فقط در معرض انواع بحث‌های سخت‌گیرانه اعمال شده در سایر حوزه‌های پیچیده حقوق آمریکا قرار دهند، می‌پذیرند.
همانطور که پیداست، قضات آمریکایی که به مسائل اراده آزاد و جبرگرایی می پردازند تقریباً همیشه به این اذعان می‌پردازند که نتایج آنها بر اساس اصول و مفروضاتی استوار است که ممکن است با پیشرفت‌های علمی و برداشت‌های فلسفی از حقیقت مرتبط نباشد و البته شاید نیازی هم ندارد.

اگرچه شاید از برخی جهات رضایت بخش نباشد، اما این رویکرد به سؤالات اراده آزاد ممکن است به طور متناوب نشان دهنده‌ی عنصر تازه‌ای از صراحت و فروتنی از سوی سیستم عدالت کیفری و قضاتی باشد که مجازات را تعیین می‌کنند.
قضاوت دیگران یک تعقیب مخاطره آمیز است، اگر اجتناب ناپذیر باشد، به ویژه در زمینه قوانین کیفری، که مجرمیت اخلاقی را به کسانی که گناهشان ثابت شده است نسبت می‌دهد.
اگرچه ممکن است مجرمان اغلب مستحق محکومیت اخلاقی باشند، ارزیابی کامل و دقیق ارزش اخلاقی یک فرد خارج از قلمرو اجرای عدالت انسانی و فراتر از درک توانایی‌های انسانی محدود باقی می‌ماند.
در اینجا نیز، اندیشه‌ی یهودی مدت‌هاست تصدیق کرده‌است که علیرغم نیاز جامعه به حفظ نظم از طریق اجرای قواعد و پیامدهای قانونی، قضاوت اخلاقی نهایی برای ولایت خداوند محفوظ است.

امتناع قضات آمریکایی از اتخاذ رویکردهای فلسفی یا علمی در قبال جبرگرایی و اراده آزاد ممکن است موجب پایبندی اساسی به استقلال قانون به عنوان نماینده ارزش‌ها و باورهای جامعه شود.
برای اطمینان، قانون باید در نظر داشته‌باشد و در صورت اقتضا، باید از پیشرفت‌های درک بشری برای اطلاع‌رسانی و بهبود عملکرد یک سیستم حقوقی استفاده کند.
در حالت ایده آل، قانون در کنار ظهور پیشرفت علمی پیشرفت خواهد کرد.
با این حال، این قانون به طور جدایی‌ناپذیری با جامعه مرتبط است و منعکس‌کننده ماهیت انسان است، که اغلب ناتوانی در مهار اکتشافات علمی را به گونه‌ای نشان داده است که ارزش‌های اساسی پیشرفت انسانی را ترویج می‌کند.
در میان درس‌های دیگر، سوء‌استفاده قضایی از فناوری DNA به‌عنوان یادآوری وسوسه‌ها و تمایلات برای بهره‌برداری از فناوری در تعقیب و اعمال قدرت است، به نحوی که ممکن است از مرزهای اخلاق و عدالت فراتر رود.
برای پیشرفت در کنار پیشرفت‌های علمی، قانون باید تعهدی هم‌زمان با پیشرفت‌های اخلاقی جاری داشته‌باشد.
\newline
\newline

{\setstretch{0.5}
\phantomsection
\subsection*{اخلاق دئونتولوژیک}
\label{subsec:اخلاق دئونتولوژیک}
\addcontentsline{toc}{subsection}{اخلاق دئونتولوژیک}{\protect\numberline{}}
\textbf{نوشته «کالین مارشال»}
\\\\
پرونده «مدل‌های ماشین در دادگاه» سؤالات اخلاقی مختلفی را از دیدگاه ریشه‌شناسی مطرح می‌کند.
دو اقدام مربوط به فناوری، به‌ویژه، مستلزم یک تحلیل ریشه‌شناختی است: (1) دادستان منطقه «فیتزپاتریک» به دنبال نتیجه‌ای متفاوت از آنچه توسط \textenglish{\textbf{TrueAllele}} ارائه شده‌بود (به شرطی که قصد او این بوده باشد) و (2) تحلیل‌گری که از \textenglish{\textbf{STRmix}} در حین ارجاع به نمایه هیلاری استفاده می‌کند (به عنوان مظنون شناخته شده پرونده).
}

یکی از تمرکزهای سنتی در اخلاق دئونتولوژیک بر رد اشکال مشکل ساز جانِبداری بوده است.
اقدامات جزئی مشکل ساز باعث مزیت نامناسب برخی افراد نسبت به دیگران می‌شود.
کسی را تصور کنید که در حال فکر کردن است که آیا یک سوار آزاد باشد یا خیر، یعنی به این فکر می‌کند که آیا از همکاری دیگران در یک سیستم سود می‌برد در حالی که خودشان همکاری نمی‌کنند.
نمونه‌هایی از سواری رایگان شامل استفاده از حمل و نقل عمومی بدون پرداخت کرایه و استفاده از خدمات دولتی و اجتناب از پرداخت مالیات است.
چنین اقداماتی به علاقه آزاد سوار بر دیگران امتیاز می‌دهد، و بنابراین (مگر اینکه عوامل کاهش دهنده وجود داشته باشد) جانبداری نامناسب را نشان می‌دهد.

در حالی که هیچ سواریِ رایگانی در مدل‌های ماشین در پرونده دادگاه رخ نمی‌دهد، ما همچنان می‌توانیم بپرسیم که آیا اقدامات (1) و (2) همانطور که توضیح داده شد، جزئی بودن مشکل‌ساز را نشان می‌دهند یا خیر؟ متأسفانه، هیچ راه کاملاً دقیق و بدون مناقشه‌ای برای شناسایی جایی که جانبداری وجود دارد یا زمانی که مشکل‌ساز است وجود ندارد.
با این حال، بسیاری از متخصصان اخلاق \textenglish{\textbf{deontology}} استفاده از روشی به نام «آزمون جهانی‌سازی» را مفید دانسته‌اند.

ایده اصلی پشت «آزمون جهانی‌سازی» یک ایده‌ی آشنا است و در این سؤال منعکس شده است: "چه می شود اگر همه این کار را انجام دهند؟".
کمی دقیق‌تر، «آزمون جهانی‌سازی» به‌صورت زیر اجرا می‌شود: یک عامل از نظر اخلاقی یک عمل احتمالی را با این سؤال از خود ارزیابی می‌کند که آیا سیستمی را تأیید می‌کند که در آن همه عوامل در موقعیت‌های مشابه به طور مشابه عمل کنند.
به عنوان مثال، فروشنده‌ای که تصمیم به دروغ گفتن به منظور تضمین یک قرارداد پرسود دارد، ممکن است در نظر داشته‌باشد که آیا مایل است سیستمی را تأیید کند که در آن همه فروشندگان به منظور تضمین قراردادهای پرسود دروغ بگویند.
در چنین سیستمی، فروشندگان عموماً غیرقابل اعتماد شناخته می‌شوند.
بنابراین چنین دروغ‌هایی در آزمون جهانی‌سازی مردود می‌شوند.
اگرچه «آزمون جهانی‌سازی» برای یک سناریوی خیالی جذاب است، اما به آشکار شدن جانبداری واقعی فروشنده در پشت دروغ کمک می‌کند و به طور نامناسبی منافع خود را بر دیگران برتری می‌دهد.

چگونه «آزمون جهانی‌سازی» برای اقدام (1) اعمال می‌شود؟ سؤالی که «دی فیتزپاتریک \textenglish{\textbf{\mbox{(DA Fitzpatrick)}}}» باید از خود می‌پرسید چیزی شبیه به این بود: آیا او سیستمی را تایید می‌کرد که در آن «وکلای دادگستری» همیشه به دنبال منبع تکنولوژیکی دیگری برای حمایت از دیدگاه‌قبلی خود بودند، در صورتی که اولین منبع چنین نبود؟ پاسخ این سؤال نسبت به فروشنده دروغگو کمتر واضح است.
با این حال، اگر همیشه (یا تقریباً همیشه) امکان یافتن منبعی تکنولوژیکی وجود داشته‌باشد که از هر حکم مورد نظر پشتیبانی می‌کند، مشکل مشابهی پیش می‌آید: در چنین سیستمی، هر گونه توسل به یک منبع ارزش متقاعدکننده خود را از دست می‌دهد.
هر کس که تلاش می‌کند به یک منبع تکنولوژیکی خاص متوسل شود، نمی تواند آن سیستم کلی را تایید کند (البته منظور بیشتر این است که هیچ کس نمی‌تواند).
غیرقابل‌قبول بودن این سناریوی خیالی نشان می‌دهد که اقدام «دی فیتزپاتریک» نشان دهنده‌ی جانبداری مشکل‌ساز است.
از سوی دیگر، اگر همیشه (یا تقریباً همیشه) امکان یافتن یک منبع فن‌آوری که هر حکم مورد نظر را پشتیبانی می‌کند ممکن نباشد، چنین سیستمی ممکن است مشکل‌ساز نباشد، که نشان می‌دهد «دی فیتزپاتریک» جانبداری مشکل‌ساز نشان نداده است.

چگونه آزمون جهانی‌سازی برای اقدام (2) اعمال می‌شود؟ در اینجا، سؤالی که تحلیلگر باید مطرح می‌کرد، در این راستا بود: آیا آن‌ها سیستمی را تأیید می‌کنند که در آن، کاربرد الگوریتم‌ها در ارزیابی احساس گناه همیشه (یا تقریباً همیشه) از باورها و سوء ظن‌های پیشین تحلیلگر مطلع باشد؟ با توجه به نقش بزرگی که تحلیلگران در کاربرد الگوریتم‌ها دارند، این تهدیدی است که جذابیت فناوری مانند \textenglish{\textbf{STRmix}} را کم ارزش می‌کند و بنابراین پشتیبانی از اتهامات نادرست را آسان می‌کند.
احتمالاً هیچ کس نمی‌تواند سیستمی را تأیید کند که در آن هرگونه اتهام نادرست با استفاده از فناوری به راحتی قابل پشتیبانی باشد.
این نشان می‌دهد که تحلیلگر در اقدام (2) جانبداری غیرقابل‌قبول از خود نشان می‌دهد.

در حالی که آزمون جهانی‌سازی در ارزیابی جزئی بودن، مفید است، نمی‌توان آن را به صورت الگوریتمی اعمال کرد.
هنگامی که آزمون برای یک عمل معین اعمال می‌شود، سؤال اصلی همیشه این است که کدام جنبه از عمل باید تعمیم داده‌شود.
به عنوان مثال، با اقدام (2)، این سؤال نباید این باشد که آیا تحلیلگر سیستمی را تأیید می‌کند که در آن هر کسی که به «نیک هیلاری» مشکوک بود، مجاز است از باورهای پیشینه خود در به کارگیری الگوریتم‌ها استفاده کند (این سؤال به شناسایی موارد مرتبط کمک نمی کند.
اشکال جانبداری در این مورد).
از این رو، آزمون همیشه باید با قضاوت‌های غیر پیش پا افتاده در مورد اینکه کدام جنبه از اعمال از نظر اخلاقی مرتبط هستند هدایت شود، و هیچ فرمول ساده ای برای تعیین اینکه آن جنبه‌ها چیست، وجود ندارد.
با این وجود، مواردی مانند فروشنده دروغگو نشان می دهد که این قضاوت‌های غیر پیش پا افتاده گاهی نسبتاً آسان و غیرقابل بحث هستند.
در حالی که اکثر مردم مستعد انواع خاصی از جانبداری هستند، با کمی فاصله، بسیاری از ما می توانیم جانبداری مشکل ساز را تشخیص دهیم.





% \textenglish{\textbf{STRmix}}

