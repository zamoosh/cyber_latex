% !TeX program = xelatex
\documentclass[12pt,oneside]{book}

\usepackage{fontspec}
\usepackage{polyglossia}
\usepackage{setspace}
\usepackage{xcolor}
\usepackage{titlesec}
\usepackage{titletoc}
\usepackage{tocloft} % Required for customizing the table of contents

% \renewcommand{\cftsecleader}{\cftdotfill{\cftdotsep}} % Adds dots between section titles and page numbers


\setmainfont[Script=Arabic,Mapping=arabicdigits]{Amiri}
\newfontfamily\arabicfont[Script=Arabic,Mapping=arabicdigits]{Amiri}

\setdefaultlanguage{persian}
\setotherlanguages{english}

\titleformat{\chapter}[display]
{\normalfont\LARGE\bfseries}{\chaptertitlename\ \thechapter}{20pt}{\Huge}
\titlespacing*{\chapter}{0pt}{-30pt}{20pt}

\titleformat{\section}
{\normalfont\Large\bfseries}{\thesection}{1em}{}

\dottedcontents{chapter}[1.5em]{\bfseries}{1.5em}{1pc}
\dottedcontents{section}[3.8em]{}{2.2em}{1pc}

\setlength\parskip{1em}

\begin{document}

    \frontmatter

    \begin{titlepage}
        \begin{center}
            \vspace*{1cm}

            {\huge \textbf{رفتار واقعی هوش‌مصنوعی برای دانشمندان داده}}

            \vspace{0.5cm}
            \LARGE




            \LARGE انتشارات

            \large تاریخ انتشار

        \end{center}
    \end{titlepage}

    \tableofcontents

    \mainmatter


    \section{تشکر نامه}
    تشکر نامه
    تشکر نامه: مایلم از مشارکت کنندگان زیر برای کمک‌های عالی و متنوعشان در این کتاب تشکر کنیم: پیتر هرشوک (اخلاق بودایی)، جان هکر رایت (اخلاق فضیلت)، ساموئل جی لوین و دانیل سینکلر (اخلاق یهودی)، کالین مارشال (اخلاق دئونتولوژیک)، جوی میلر و آندریا سالیوان کلارک (اخلاق بومی) و جان مورانگی (اخلاق آفریقایی). هدف ما ترسیم تصویری اخلاقی است که تا حد امکان منتوع و جذاب باشد. بدون کمک این عزیزان خردمند، توانستیم این کتاب را ایجاد کنیم!

    \newpage


    \section{فهرست همکاران و مشارکت کنندگان}
    \twocolumn

    \begin{flushright}
        \begin{Large}
            \textbf{پیتر سینگر}
        \end{Large}
        \\
        مرکز دانشگاهی برای ارزش های انسانی دانشگاه پرینستون
    \end{flushright}

    \begin{flushright}
        \begin{Large}
            \textbf{ییپ فای تسه}
        \end{Large}
        \\
        مرکز دانشگاهی برای ارزش های انسانی دانشگاه پرینستون
    \end{flushright}

    \begin{flushright}
        \begin{Large}
            \textbf{سامول جی لوین}
        \end{Large}
        \\
        مرکز حقوقی تورو
    \end{flushright}

    \begin{flushright}
        \begin{Large}
            \textbf{کولین مارشال}
        \end{Large}
        \\
        گروه فلسفه دانشگاه واشنگتن
    \end{flushright}

    \begin{flushright}
        \begin{Large}
            \textbf{جویی میلر}
        \end{Large}
        \\
        گروه فلسفه دانشگاه وست چستر
    \end{flushright}

    \begin{flushright}
        \begin{Large}
            \textbf{جان مورونگی}
        \end{Large}
        \\
        گروه فلسفه دانشگاه وست چستر
    \end{flushright}


    \chapter{فصل اول}

    این یک مثال است.


    \section{بخش اول}\label{sec:-}

    این یک مثال است.

    \subsection{زیر بخش اول}

    این یک مثال است.


    \section{بخش دوم}

    این یک مثال است.

    \backmatter

\end{document}
